\documentclass[a4]{article}

\usepackage[left=2cm,right=2cm,top=2cm,bottom=2cm]{geometry} 

\usepackage[utf8]{inputenc}   % otra alternativa para los caracteres acentuados y la "ñ"
\usepackage[           spanish % para poder usar el español
                      ,es-tabla % para los captions de las tablas
                       ]{babel}   
\decimalpoint %para usar el punto decimal en vez de coma para los números con decimales

\usepackage{beton}
\usepackage[T1]{fontenc}

\usepackage{parskip}
\usepackage{xcolor}

\usepackage{marvosym}

\usepackage{caption}

\usepackage{enumerate} % paquete para poder personalizar fácilmente la apariencia de las listas enumerativas

\usepackage{graphicx} % figuras
\usepackage{subfigure} % subfiguras

\usepackage{amsfonts}
\usepackage{amsmath}

\usepackage{chngcntr}

\counterwithin*{equation}{section}

\definecolor{gris}{RGB}{220,220,220}
	
\usepackage{float} % para controlar la situación de los entornos flotantes

\restylefloat{figure}
\restylefloat{table} 
\setlength{\parindent}{0mm}


\usepackage[bookmarks=true,
            bookmarksnumbered=false, % true means bookmarks in 
                                     % left window are numbered
            bookmarksopen=false,     % true means only level 1
                                     % are displayed.
            colorlinks=true,
            allcolors=blue]{hyperref}
\definecolor{webblue}{rgb}{0, 0, 0.5}  % less intense blue


\title{Ejercicios de la ecuación del calor y programación lineal}

\author{David Cabezas Berrido}

\date{\vspace{-4mm}}

\begin{document}
\maketitle

\section{Ecuación del calor}

Debemos encontrar una solución del problema:

\begin{equation} \label{eq:enunciado}
  \begin{cases}
    u_t=u_{xx}, \qquad (t,x)\in [0,+\infty)\times [0,\pi] \\
    u(t,0)=0,\quad u(t,\pi)=\pi, \qquad t\in [0,+\infty) \\
    u(0,x)=x+\sin(x), \qquad x\in [0,\pi]
  \end{cases}
\end{equation}

y estudiar entre qué clase de funciones podríamos afirmar que es
única.

\vspace{4mm}

Consideramos primero el problema

\begin{equation} \label{eq:homogeneo}
  \begin{cases}
    u_t=u_{xx}, \qquad (t,x)\in [0,+\infty)\times [0,\pi] \\
    u(t,0)=u(t,\pi)=0   , \qquad t\in [0,+\infty) \\
    u(0,x)=\sin(x), \qquad x\in [0,\pi]
  \end{cases}
\end{equation}

Este es el problema que hemos estudiado en teoría (con
$\varphi(x)=\sin(x)\quad\forall x\in[0,\pi]$) y conocemos que la
solución es
\begin{equation} \label{eq:sol-homogeneo}
  u(t,x)=\sum_{n=1}^\infty a_n e^{-n^2 t}\sin(nx)\qquad\forall(t,x)\in[0,+\infty)\times [0,\pi]
\end{equation}
donde
\begin{equation} \label{eq:an}
  a_n=\frac{2}{\pi}\int_0^\pi\sin(x)\sin(nx)dx\qquad\forall n\in\mathbb{N}
\end{equation}
Para $n>1$, tenemos
\begin{align*}
  \frac{\pi}{2}a_n&=\int_0^\pi\sin(x)\sin(nx)dx=-\sin(nx)\cos(x)\bigg]_0^\pi+n\int_0^\pi\cos(x)\cos(nx)dx=\\&= 0+n\bigg(\sin(x)\cos(nx)\bigg]_0^\pi+n\int_0^\pi\sin(x)\sin(nx)dx\bigg)=0+n^2\frac{\pi}{2}a_n \Longrightarrow a_n=0
\end{align*}
Para $n=1$, tenemos
\[a_1=\frac{2}{\pi}\int_0^\pi\sin^2(x)dx=\frac{2}{\pi}\int_0^\pi\frac{1-\cos(2x)}{2}dx=\frac{2}{\pi}\frac{\pi}{2}-\frac{2}{\pi}\int_0^\pi\frac{\cos(2x)}{2}dx=1-\frac{2}{\pi}\bigg[\frac{\sin(2x)}{4}\bigg]_{0}^{\pi}=1\]

No debe sorprendernos que el desarollo en serie de Fourier de
$\sin(x)$ sea $\sin(x)$.

Por tanto la solución de (\ref{eq:homogeneo}) es
\begin{equation} \label{eq:u}
  u(t,x)=e^{-t}\sin(x)\qquad\forall(t,x)\in[0,+\infty)\times [0,\pi]
\end{equation}
Se comprueba rápidamente que $u_t=-u=u_{xx}$, $u(t,0)=u(t,\pi)=0$ y
$u(0,x)=\sin(x)$.

\newpage

Tomamos ahora $v(t,x)=x+u(t,x)=x+e^{-t}\sin(x)\qquad\forall(t,x)\in[0,+\infty)\times [0,\pi]$

Tenemos $v_t=u_t$ y $v_{xx}=u_{xx}$, por lo que $v$ satisface la
ecuación diferencial de (\ref{eq:enunciado}).

También, $v(t,0)=0+u(t,0)=0$ y $v(t,\pi)=\pi+u(t,\pi)=\pi$ para todo
$t$ en $\mathbb{R}^+_0$. Y $v(0,x)=x+u(0,x)=x+\sin(x)$ para todo $x$ en
$[0,\pi]$.

Por tanto $v\in C^2\big([0,+\infty)\times[0,\pi]\big)$ es solución de (\ref{eq:enunciado}).

\vspace{4mm}

Ahora supongamos que
$w\in C^2\big((0,+\infty)\times[0,\pi]\big)\cap
C\big([0,+\infty)\times[0,\pi]\big)$ es otra solución de
(\ref{eq:enunciado}), por tanto tendremos $w_t=w_{xx}$, $w(t,0)=0$ y
$w(t,\pi)=\pi$ para todo $t$ en $\mathbb{R}^+$. Y $w(0,x)=x+\sin(x)$
para todo $x$ en $[0,\pi]$.

Tomando $u=v-w$ obtenemos $u_t=u_{xx}$, $u(t,0)=u(t,\pi)=0$ para todo
$t$ en $\mathbb{R}^+_0$ y $u(0,x)=0$ para todo $x$ en $[0,\pi]$.

Por tanto $u\in C^2\big((0,+\infty)\times[0,\pi]\big)\cap
C\big([0,+\infty)\times[0,\pi]\big)$ es solución del
problema ($\varphi(x)=0\quad\forall x\in[0,\pi]$).

\begin{equation} \label{eq:nulo}
  \begin{cases}
    u_t=u_{xx}, \qquad (t,x)\in [0,+\infty)\times [0,\pi] \\
    u(t,0)=u(t,\pi)=0, \qquad t\in [0,+\infty) \\
    u(0,x)=0, \qquad x\in [0,\pi]
  \end{cases}
\end{equation}

El problema satisface claramente las hipótesis del teorema de
unicidad, luego tiene (sabemos que la tiene porque $u$ lo satisface)
solución única en $C\big([0,+\infty)\times[0,\pi]\big)$. Como la
función constante 0 en $[0,+\infty)\times[0,\pi]$ también es solución,
$u=0$. Por tanto $v=w$ y afirmamos que $v$ es la única solución de
(\ref{eq:enunciado}) en $C\big([0,+\infty)\times[0,\pi]\big)$.

Ahora probaré que la solución también es única en el sentido
$L^2$. Sea $w\in C^2\big((0,+\infty)\times[0,\pi]\big)$ tal que \\
$w(t,\cdot)\in L^2[0,\pi]$ para todo $t$ cercano a 0
($t\in (0,\varepsilon)$ con $\varepsilon>0$) y (llamando
$g(x)=x+\sin(x)$)
\[\lim_{t\to 0^+}\big\|w(t,\cdot)-g\big\|_{L^2}=0\]
también $w_t=w_{xx}$ en $(0,+\infty)\times[0,\pi]$ y $w(t,0)=0$,
$w(t,\pi)=\pi$ para todo $t$ en $\mathbb{R}^+$. Probaré que $w=v$ en
$(0,+\infty)\times[0,\pi]$. Tomando
$u=w-v\in C^2\big((0,+\infty)\times[0,\pi]\big)$,
$u(t,\cdot)\in L^2[0,\pi]\quad\forall t\in (0,\varepsilon)$, tenemos

$u_t=w_t-v_t=w_{xx}-v_{xx}=u_{xx}$ en $(0,+\infty)\times[0,\pi]$ \\
$u(t,0)=w(t,0)-v(t,0)=0-0=0$ para todo $t$ en $\mathbb{R}^+$ \\
$u(t,\pi)=w(t,\pi)-v(t,\pi)=\pi-\pi=0$ para todo $t$ en $\mathbb{R}^+$
\[\lim\limits_{t\to 0^+}\big\|u(t,\cdot)\big\|_{L^2}=\lim\limits_{t\to 0^+}\big\|w(t,\cdot)-v(t,\cdot)\big\|_{L^2}\leq \lim\limits_{t\to 0^+}\big\|w(t,\cdot)-g\big\|_{L^2}+\lim\limits_{t\to 0^+}\big\|v(t,\cdot)-g\big\|_{L^2}=0+0=0\]
Por tanto $u$ es solución en el sentido $L^2$ del problema
(\ref{eq:nulo}). La función constante 0 $(0,+\infty)\times[0,\pi]$
también es solución del problema en el sentido $L^2$, y aplicando la
proposición de unicidad en sentido $L^2$ que hemos estudiado en teoría
obtenemos $u=0$ y $v=w$ como queríamos. Por lo que $v$ también es
única en sentido $L^2$.

\newpage

\section{Programación lineal}

Una fábrica envasadora de alimentos recibe diariamente
\begin{itemize}
\item 1500 kg de café tipo A.
\item 715 kg de café tipo B.
\item 190 kg de café tipo C.
\end{itemize}
con ellos fabrica tres mezclas
\begin{itemize}
\item Mezcla tipo 1: 80\% café A y 20\% café B; beneficio de 0.2
  \EUR/kg.
\item Mezcla tipo 2: 60\% café B y 40\% café C; beneficio de 0.3
  \EUR/kg.
\item Mezcla tipo 3: 40\% café A, 44\% café B y 16\% café C; beneficio
  de 0.4 \EUR/kg.
\end{itemize}

Queremos conocer la cantidad de mezcla de cada tipo que debemos
producir para maximizar la ganancia.

\vspace{4mm}

Si denotamos $x_1$, $x_2$ y $x_3$ a la cantidad en kg que fabricamos
de mezcla de tipo 1, 2 y 3 respectivamente, el beneficio en \EUR\
(función a maximizar) es
\begin{equation} \label{eq:beneficio}
BE=0.2x_1+0.3x_2+0.4x_3
\end{equation}
Hay ciertas restricciones, pues tenemos recursos limitados. La
cantidad total en kg de café de tipo A utilizado es $0.8x_1+0.4x_3$ y
no podrá exceder los 1500 kg. Del mismo modo la cantidad de tipo B
($0.2x_1+0.6x_2+0.44x_3$) no podrá exceder 715 kg y la de tipo C
($0.4x_2+0.16x_3$) los 190 kg. Por tanto las restricciones que
encontramos son
\begin{equation} \label{eq:restricciones}
  \begin{split} 
  0.8x_1+0.4x_3 \leq 1500 \\
  0.2x_1+0.6x_2+0.44x_3 \leq 715 \\
  0.4x_2+0.16x_3 \leq 190 \\
\end{split} \ \Longleftrightarrow
\begin{pmatrix}
    0.8 & 0 & 0.4 \\
    0.2 & 0.6 & 0.44 \\
    0 & 0.4 & 0.16
  \end{pmatrix}
  \begin{pmatrix}
    x_1 \\ x_2 \\ x_3
  \end{pmatrix} \leq
  \begin{pmatrix}
    1500 \\ 715 \\ 190
  \end{pmatrix}
\end{equation}
Por tanto problema de gestión de recursos que debemos resolver es
\begin{equation} \label{eq:prob-recursos}
  \begin{cases}
    Ax\leq b \\
    x\leq 0 \\
    \max f(x)=<p,x>+m
  \end{cases}
\end{equation}
Donde
\[A=\begin{pmatrix}
    0.8 & 0 & 0.4 \\
    0.2 & 0.6 & 0.44 \\
    0 & 0.4 & 0.16
  \end{pmatrix}\quad b=\begin{pmatrix}
    1500 \\ 715 \\ 190
  \end{pmatrix}\quad x=\begin{pmatrix}
    x_1 \\ x_2 \\ x_3
  \end{pmatrix}\quad p=\begin{pmatrix}
    0.2 \\ 0.3 \\ 0.4
  \end{pmatrix}\quad m=0
\]

Llamando $h=(h_1, h_2, h_3)$ al vector de holguras, obtenemos el
problema estándar asociado
\begin{equation} \label{eq:prob-estandar}
  \begin{cases}
    Ax+h= b \\
    x\geq 0, \quad h\geq 0 \\
    \max f(x)=<p,x>+m
  \end{cases}
\end{equation}
Lo resolveremos por el método Simplex. Como no nos dicen nada,
suponemos que el café sin mezclar no aporta ningún beneficio a la
fábrica envasadora, luego en la matriz ampliada las holguras tendrán
ceros en la última fila. La matriz ampliada queda de esta forma.
\[\begin{pmatrix}
    x_1 & x_2 & x_3 & h_1 & h_2 & h_3 & b \\
    0.8 & 0 & 0.4 & 1 & 0 & 0 & 1500 \\
    0.2 & 0.6 & 0.44 & 0 & 1 & 0 & 715 \\
    0 & 0.4 & 0.16 & 0 & 0 & 1 & 190 \\
    0.2 & 0.3 & 0.4 & 0 & 0 & 0 & f-0
  \end{pmatrix}\] Primero elegimos la primera columna para
pivotar. Dividiendo la columna de $b$ entre la de $x_1$ (sólo las dos
primeras filas puesto que sólo consideramos entradas positivas y hay
un $0$ en la tercera fila) obtenemos $\frac{1500}{0.8}=1875$,
$\frac{715}{0.2}=3575$ Para evitar obtener valores negativos en la
columna $b$ , hacemos un $1$ en la primera fila (menor de los
cocientes), para ello la dividimos entre $0.8$. En las demás filas
haremos un $0$ en la primera columna, restandoles la primera fila (ya
con un $1$ en esa columna) multiplicada por el valor correspondiente a
$x_1$ en esa fila. Obtenemos
\[\begin{pmatrix}
    x_1 & x_2 & x_3 & h_1 & h_2 & h_3 & b \\
    1 & 0   & 0.5  & 1.25  & 0 & 0 & 1875 \\
    0 & 0.6 & 0.34 & -0.25 & 1  & 0 & 340 \\
    0 & 0.4 & 0.16 & 0     & 0  & 1 & 190 \\
    0 & 0.3 & 0.3  & -0.25 & 0 & 0 & f-375
  \end{pmatrix}\]
Ahora pivotamos sobre la segunda columna (debemos
siempre elegir una con entrada positiva en la fila correspondiente a
la función objetivo, la última). Para decidir si pivotar sobre las
filas 2 ó 3 (la 1 tiene un 0, elegimos entradas positivas) calculamos
los cocientes $\frac{340}{0.6}=566.\overline{6}$ y 
$\frac{190}{0.4}=475$; elegimos el menor de ellos, luego pivotaremos
sobre la tercera fila. Obtenemos
\[\begin{pmatrix}
    x_1 & x_2 & x_3 & h_1 & h_2 & h_3 & b \\
    1 & 0 & 0.5  & 1.25  & 0 &   0   & 1875 \\
    0 & 0 & 0.1  & -0.25 & 1 & -1.5  & 55 \\
    0 & 1 & 0.4  &   0   & 0 &  2.5  & 475 \\
    0 & 0 & 0.18 & -0.25 & 0 & -0.75 & f-517.5
  \end{pmatrix}\]

Ahora pivotamos sobre la columna 3, la única
posible. $\frac{1875}{0.5}=3750$, $\frac{55}{0.1}=550$ y
$\frac{475}{0.4}=1187.5$; así que pivotamos sobre la segunda fila
obteniendo
\[\begin{pmatrix}
    x_1 & x_2 & x_3 & h_1 & h_2 & h_3 & b \\
    1 & 0 & 0 &  2.5 &  -5  &  7.5  & 1600 \\
    0 & 0 & 1 & -2.5 &  10  &  -15  & 550 \\
    0 & 1 & 0 &   1  &  -4  &  8.5  & 255 \\
    0 & 0 & 0 &  0.2 & -1.8 &  1.95 & f-616.5
  \end{pmatrix}\] Ahora pivotamos sobre la columna
4. $\frac{1600}{2.5}=640$, $\frac{255}{1}=255$. Fila 3. Obtenemos
\[\begin{pmatrix}
    x_1 & x_2 & x_3 & h_1 & h_2 & h_3 & b \\
    1 & -2.5 & 0 & 0 &  5 & -13.75 & 962.5 \\
    0 &  2.5 & 1 & 0 &  0 &  6.25  & 1187.5 \\
    0 &   1  & 0 & 1 & -4 &  8.5   & 255 \\
    0 & -0.2 & 0 & 0 & -1 &  0.25  & f-667.5
  \end{pmatrix}\] Pivotamos sobre la columna
6. $\frac{1187.5}{6.25}=190$, $\frac{255}{8.5}=30$. Fila 3. Obtenemos
\[\begin{pmatrix}
    x_1 & x_2 & x_3 & h_1 & h_2 & h_3 & b \\
    1 & -0.8824 & 0 &  1.6176 & -1.4706 & 0 & 1375 \\
    0 &  1.7647 & 1 & -0.7353 &  2.9412 & 0 & 1000 \\
    0 &  0.1176 & 0 &  0.1176 & -0.4756 & 1 & 30 \\
    0 & -0.2294 & 0 & -0.2294 & -0.8824 & 0 & f-675
  \end{pmatrix}\]

No queda ningún peso positivo en la última fila, luego hemos llegado a
una tabla terminal. Las tres variables básicas que hemos obtenido son
$x_1$, $x_3$ y $h_3$. Mirando la posición de los 1$_\text{s}$,
concluimos que para maximizar el beneficio se deben fabricar
\textbf{1375 kg de la mezcla 1, 1000 kg de la mezcla 3 y 0 kg de la
  mezcla 2}. Sobrando 30 kg de café tipo C.

La cantidad de café A que se gasta es
$0.8\cdot 1375+0.4\cdot 1000=1500$ kg, de café B se gastan
$0.2\cdot 1375+0.6\cdot 0+0.44\cdot 1000=715$ kg y de café C se gastan
$0.4\cdot 0+0.16\cdot 1000=160$ kg (sobran 30 kg como sabíamos).

Fabricando estas cantidades, se obtiene un beneficio de
$0.2\cdot 1375+0.3\cdot 0+0.4\cdot 1000=675$ \EUR

\end{document}
