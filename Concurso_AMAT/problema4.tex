\documentclass[12pt,spanish]{article}
% aprovechamiento de la p\'agina -- fill an A4 (210mm x 297mm) page
% Note: 1 inch = 25.4 mm = 72.27 pt
% 1 pt = 3.5 mm (approx)

% vertical page layout -- one inch margin top and bottom
\topmargin      -10 mm   % top margin less 1 inch
\headheight       0 mm   % height of box containing the head
\headsep          0 mm   % space between the head and the body of the page
\textheight     255 mm   % the height of text on the page
\footskip         7 mm   % distance from bottom of body to bottom of foot

% horizontal page layout -- one inch margin each side
\oddsidemargin    0 mm     % inner margin less one inch on odd pages
\evensidemargin   0 mm     % inner margin less one inch on even pages
\textwidth      159 mm     % normal width of text on page

\setlength{\parindent}{0pt}

\usepackage{tikz}
\usetikzlibrary{automata,positioning}

\usepackage[doument]{ragged2e}
\usepackage{babel}
\usepackage[utf8]{inputenc}
\usepackage{amsmath,amsthm,mathtools}
\usepackage{amsfonts,amssymb,latexsym}
\usepackage{enumerate}
\usepackage{subfigure, float, graphicx, caption}
\captionsetup[table]{labelformat=empty}
\captionsetup[figure]{labelformat=empty}
\definecolor{RojoAnayelRey}{rgb}{1,.25,.25}
\usepackage[bookmarks=true,
            bookmarksnumbered=false, % true means bookmarks in 
                                     % left window are numbered                         
            bookmarksopen=false,     % true means only level 1
                                     % are displayed.
            colorlinks=true,
            linkcolor=webred]{hyperref}
\definecolor{webgreen}{rgb}{0, 0.5, 0} % less intense green
\definecolor{webblue}{rgb}{0, 0, 0.5}  % less intense blue
\definecolor{webred}{rgb}{0.5, 0, 0}   % less intense red
\definecolor{dkgreen}{rgb}{0,0.6,0}
\definecolor{gray}{rgb}{0.5,0.5,0.5}
\definecolor{mauve}{rgb}{0.58,0,0.82}
\definecolor{MistyRose}{RGB}{255,228,225}
\definecolor{LightCyan}{RGB}{224,255,255}

\begin{document}

\title{Solución al problema 1: Dilema del padre}
\author{David Cabezas Berrido}
\date{\vspace{-5mm}}
\maketitle

\section{a) Encontrar, si existe, un entero positivo cuyo factorial termine en exactamente 5 ceros.}

Entiendo que esto significa que la cifra correspondiente a las
unidades de millón sea distinta de 0 y todas las cifras anteriores
(desde las unidades a las centenas de millar) sean 0. \\

No existe tal número, a continuación lo razonaré: \\

Para que un número termine en exactamente 5 0s, debe ser divisible por
$10^5$, pero no por $10^6$, esto significa que en su factorización en
números primos aparezcan $2^p\cdot 5^q$ con $p,q \in \mathbb{N}$ y min$\{p,q\}=5$. \\

Si un número $n$ es de la forma $n=k!$, la multiplicidad del factor 2
(aparece cada dos números) será menor que la multiplicidad del factor
5 (aparece cada 5), por lo que podemos suponer que si $k$ es el número
buscado, la factorización de $n$ en números primos tendrá
$2^p\cdot 5^5$ con $p>5$. Pero esto es imposible, ya que si $k=24$, el
factor 5 tendrá multiplicidad 4 (aparece en 5, 10, 15, 20); y para
$k=25$, el factor 5 tendrá multiplicidad 6 (aparece dos veces en 25). \\

Por lo que no existe ningún número cuyo factorial tenga al 5 como
factor de multiplicidad exactamente 5, y puesto que el factor 2
aparece con mayor multiplicidad, acabe en 5 0s.
\end{document}