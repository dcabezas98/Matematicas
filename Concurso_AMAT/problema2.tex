\documentclass[12pt,spanish]{article}
% aprovechamiento de la p\'agina -- fill an A4 (210mm x 297mm) page
% Note: 1 inch = 25.4 mm = 72.27 pt
% 1 pt = 3.5 mm (approx)

% vertical page layout -- one inch margin top and bottom
\topmargin      -10 mm   % top margin less 1 inch
\headheight       0 mm   % height of box containing the head
\headsep          0 mm   % space between the head and the body of the page
\textheight     255 mm   % the height of text on the page
\footskip         7 mm   % distance from bottom of body to bottom of foot

% horizontal page layout -- one inch margin each side
\oddsidemargin    0 mm     % inner margin less one inch on odd pages
\evensidemargin   0 mm     % inner margin less one inch on even pages
\textwidth      159 mm     % normal width of text on page

\setlength{\parindent}{0pt}

\usepackage{tikz}
\usetikzlibrary{automata,positioning}

\usepackage[doument]{ragged2e}
\usepackage{babel}
\usepackage[utf8]{inputenc}
\usepackage{amsmath,amsthm,mathtools}
\usepackage{amsfonts,amssymb,latexsym}
\usepackage{enumerate}
\usepackage{subfigure, float, graphicx, caption}
\captionsetup[table]{labelformat=empty}
\captionsetup[figure]{labelformat=empty}
\definecolor{RojoAnayelRey}{rgb}{1,.25,.25}
\usepackage[bookmarks=true,
            bookmarksnumbered=false, % true means bookmarks in 
                                     % left window are numbered                         
            bookmarksopen=false,     % true means only level 1
                                     % are displayed.
            colorlinks=true,
            linkcolor=webred]{hyperref}
\definecolor{webgreen}{rgb}{0, 0.5, 0} % less intense green
\definecolor{webblue}{rgb}{0, 0, 0.5}  % less intense blue
\definecolor{webred}{rgb}{0.5, 0, 0}   % less intense red
\definecolor{dkgreen}{rgb}{0,0.6,0}
\definecolor{gray}{rgb}{0.5,0.5,0.5}
\definecolor{mauve}{rgb}{0.58,0,0.82}
\definecolor{MistyRose}{RGB}{255,228,225}
\definecolor{LightCyan}{RGB}{224,255,255}

\begin{document}

\title{Solución al problema 2: Roscón Criptoanalítico}
\author{David Cabezas Berrido}
\date{\vspace{-5mm}}
\maketitle

Queremos descifrar el siguiente mensaje:

\begin{center}
  Dqxwpcmadqc, xcr pdradñsw dñ epwmñdoc. \\
  Drjphmd odñwjwswq dq dñ craqsw fdñ jwppdw ecpc rcmdp gad ñw shdqdr mhdq.
\end{center}

El cifrado es una biyección del alfabeto a él mismo, sospecho esto
porque \textit{dñ} y \textit{fdñ} tienen pinta de ser \textit{el} y \textit{del}.

He realizado un programa en \texttt{Python} para probar combinaciones
y ayudarme a encontrar la solución:

\begin{verbatim}
enigma=list("""dqxwpcmadqc, xcr pdradñsw dñ epwmñdoc.
drjphmd odñwjwswq dq dñ craqsw fdñ jwppdw ecpc rcmdp gad ñw shdqdr mhdq.""")

solucion=enigma.copy()

print(''.join(solucion), '\n')

# Es muy probable que 'dñ' y 'fdñ' sean 'EL' y 'DEL'.
# Esto nos sugiere que la encriptación es una biyección del alfabeto sobre él mismo

# Función para cambiar letras del mensaje original (las nuevas las pongo en mayúscula para diferenciar
def cambio(a, b):
    for i in range(len(enigma)):
        if enigma[i]==a:
            solucion[i]=b.upper()

cambio('d','e')
cambio('ñ','l')
cambio('f','d')
print(''.join(solucion), '\n')

"""
EqxwpcmaEqc, xcr pEraELsw EL epwmLEoc.
ErjphmE oELwjwswq Eq EL craqsw DEL jwppEw ecpc rcmEp gaE Lw shEqEr mhEq.
"""

# 'ecpc' tiene que ser una palabra del tipo 'poco', 'como', 'PARA'

cambio('e','p')
cambio('c','a')
cambio('p','r')
print(''.join(solucion), '\n')

"""
EqxwRAmaEqA, xAr REraELsw EL PRwmLEoA.
ErjRhmE oELwjwswq Eq EL Araqsw DEL jwRREw PARA rAmER gaE Lw shEqEr mhEq.
"""
# 'rAmER' será un verbo como 'lamer' (pero la L ya es la ñ) o 'SABER

cambio('r','s')
cambio('m','b')
print(''.join(solucion), '\n')

"""
EqxwRABaEqA, xAS RESaELsw EL PRwBLEoA.
ESjRhBE oELwjwswq Eq EL ASaqsw DEL jwRREw PARA SABER gaE Lw shEqES BhEq.
"""

# 'PRwBLEoA' suena a 'PROBLEMA'

cambio('w','o')
cambio('o','m')
print(''.join(solucion), '\n')

"""
EqxORABaEqA, xAS RESaELsO EL PROBLEMA.
ESjRhBE MELOjOsOq Eq EL ASaqsO DEL jORREO PARA SABER gaE LO shEqES BhEq.
"""

# 'jORREO' a 'CORREO' y 'xAS' a 'HAS'

cambio('j','c')
cambio('x','h')
print(''.join(solucion), '\n')

"""
EqHORABaEqA, HAS RESaELsO EL PROBLEMA.
ESCRhBE MELOCOsOq Eq EL ASaqsO DEL CORREO PARA SABER gaE LO shEqES BhEq.
"""

# 'RESaELsO' a 'RESUELTO'

cambio('a','u')
cambio('s','t')
print(''.join(solucion), '\n')

"""
EqHORABUEqA, HAS RESUELTO EL PROBLEMA.
ESCRhBE MELOCOTOq Eq EL ASUqTO DEL CORREO PARA SABER gUE LO ThEqES BhEq.
"""


# 'EqHORABUEqA' a 'ENHORABUENA', 'ESCRhBE' a 'ESCRIBE', 'gUE' a 'QUE'
cambio('g','q')
cambio('q','n')
cambio('h','i')
print(''.join(solucion), '\n')

"""
ENHORABUENA, HAS RESUELTO EL PROBLEMA.
ESCRIBE MELOCOTON EN EL ASUNTO DEL CORREO PARA SABER QUE LO TIENES BIEN.
"""
\end{verbatim}

\end{document}