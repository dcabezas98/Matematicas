\documentclass[12pt,spanish]{article}

% aprovechamiento de la p\'agina -- fill an A4 (210mm x 297mm) page
% Note: 1 inch = 25.4 mm = 72.27 pt
% 1 pt = 3.5 mm (approx)

% vertical page layout -- one inch margin top and bottom
\topmargin      -10 mm   % top margin less 1 inch
\headheight       0 mm   % height of box containing the head
\headsep          0 mm   % space between the head and the body of the page
\textheight     255 mm   % the height of text on the page
\footskip         7 mm   % distance from bottom of body to bottom of foot

% horizontal page layout -- one inch margin each side
\oddsidemargin    0 mm     % inner margin less one inch on odd pages
\evensidemargin   0 mm     % inner margin less one inch on even pages
\textwidth      159 mm     % normal width of text on page

\usepackage[doument]{ragged2e}
\usepackage{babel}
\usepackage[utf8]{inputenc}
\usepackage{amsmath,amsthm,mathtools}
\usepackage{amsfonts,amssymb,latexsym}
\usepackage{enumerate}
\usepackage[dvips,usenames]{color}
\definecolor{RojoAnayelRey}{rgb}{1,.25,.25}
\usepackage{tikz}
\usepackage[bookmarks=true,
            bookmarksnumbered=false, % true means bookmarks in 
                                     % left window are numbered                         
            bookmarksopen=false,     % true means only level 1
                                     % are displayed.
            colorlinks=true,
            linkcolor=webred]{hyperref}
\definecolor{webgreen}{rgb}{0, 0.5, 0} % less intense green
\definecolor{webblue}{rgb}{0, 0, 0.5}  % less intense blue
\definecolor{webred}{rgb}{0.5, 0, 0}   % less intense red
\definecolor{dkgreen}{rgb}{0,0.6,0}
\definecolor{gray}{rgb}{0.5,0.5,0.5}
\definecolor{mauve}{rgb}{0.58,0,0.82}
\definecolor{MistyRose}{RGB}{255,228,225}
\definecolor{LightCyan}{RGB}{224,255,255}

\usepackage{beton}
\usepackage[T1]{fontenc}

% Theorem environments

%% \theoremstyle{plain} %% This is the default
\newtheorem{theorem}{Teorema}[section]
\newtheorem{corollary}[theorem]{Corolario}
\newtheorem{lemma}[theorem]{Lema}
\newtheorem{proposition}[theorem]{Proposici\'on}
%\newtheorem{ax}{Axioma}

\theoremstyle{definition}
\newtheorem{definition}{Definici\'on}[section]
\newtheorem{algorithm}{\textrm{\bf Algoritmo}}[section]

\theoremstyle{remark}
\newtheorem{remark}{Observaci\'on}[section]
\newtheorem{example}{Ejemplo}[section]
\newtheorem{exercise}{Ejercicio}[section]
%\newenvironment{solution}{\begin{proof}[Solution]}{\end{proof}}
\newenvironment{solution}{\begin{proof}[Solución]}{\end{proof}}
\newtheorem*{notation}{Notaci\'on}

%\numberwithin{equation}{section}

%\newcommand{\regla}[2]{
%\begin{array}{c}
%#1\\
%\hline
%#2\\
%\end{array}
%}

\title{Problema de las n cartas introducidas al azar en n sobres}
\author{David Cabezas Berrido}
\date{}

\begin{document}

\maketitle

\begin{justify}
  Una secretaria introduce n cartas dirigidas a n destinatarios
  distintos en n sobres al azar, se desea calcular la probabilidad de
  que al menos una carta llegue a su destino.
\end{justify}

\begin{justify}
  Primero identifiquemos nuestro espacio probabilístico, los posibles
  sucesos son las diferentes formas de repartir las n cartas en n
  sobres de forma que a cada sobre le corresponda una única carta
  (obviamente no pueden quedar cartas fuera), es decir, las
  permutaciones de n elementos.
\end{justify}

\begin{align*}
  \Omega&=\mathnormal{Perm}(n)\\
  |\Omega|&=n!
 \end{align*}
    
\begin{justify}
  La probabilidad de cada permutación es $\frac{1}{n!}$, por lo tanto
  el número de casos posibles es el cardinal de $\Omega$, n!.
\end{justify}

\begin{justify}
  Los casos favorables son aquellas permutaciones en las que alguna de
  las cartas queda en su posión correcta, es decir, el complementario
  de los desbarajustes. Por tanto la probabilidad de que alguna carta
  llegue a su destino es 1 menos la probabilidad de que aparezca un
  desbarajuste. Notaremos $A$ al suceso que ocurre cuando alguna carta
  llega a su destino, nuestro objetivo es calular su probabilidad, y
  $D$ al suceso que ocurre cuando ninguna carta llega a su destino, es
  decir, se da un desbarajuste. Obviamente, $A$ y $D$ son sucesos
  complementarios.
  $P(A)=1-P(D)$
\end{justify}

\begin{align*}
  D(n)=n!\sum\limits_{i=0}^n\frac{(-1)^i}{i!}
\end{align*}

\begin{justify}
  Como todos los casos son equiprobables (cada una de las
  permutaciones en las cartas tiene la misma probabilidad de ocurrir),
  calculamos la probabilidad de que ninguna carta llegue al
  destinatario adecuado dividiendo el número de casos favorables entre
  el número de casos posibles.
\end{justify}

\begin{align*}
  P(D)&=\frac{D(n)}{n!}\\
  &=\frac{n!\sum\limits_{i=0}^n\frac{(-1)^i}{i!}}{n!}\\
  &=\sum\limits_{i=0}^n\frac{(-1)^i}{i!}\\
  &=\frac{1}{0!}-\frac{1}{1!}+\frac{1}{2!}-\frac{1}{3!}+\cdots+\frac{(-1)^n}{n!}\\
  &=\frac{1}{2!}-\frac{1}{3!}+\cdots+\frac{(-1)^n}{n!}
\end{align*}

\begin{justify}
  Por tanto, la probabilidad de que al menos una carta llegue a su
  destino, que es la que buscamos, viene dada por:
\end{justify}

\begin{align*}
  P(A)&=1-P(D)\\
  &=1-\frac{1}{2!}+\frac{1}{3!}-\frac{1}{4!}+\cdots+\frac{(-1)^{n+1}}{n!}
\end{align*}

\begin{justify}
  Esta serie es convergente, por lo que podemos calcular el límite de
  $P(A)$ en el infinito, la probabilidad de que aparezca un
  desbarajuste, $P(D)$, se corresponde con el polinomio de Taylor de
  orden n de la función $f(x)=e^x$ centrado en $0$ y evaluado en
  $x=-1$, recordemos que este polinomio es:
\end{justify}

\begin{align*}
  P_n(x)&=1+\frac{x}{1}+\frac{x^2}{2!}-\frac{1}{3!}+\cdots+\frac{(-1)^n}{n!}\\
  P_n(-1)&=1-1+\frac{1}{2!}-\frac{1}{3!}+\cdots+\frac{(-1)^n}{n!}\\
  &=P(D)
\end{align*}

\begin{justify}
  Por tanto, a medida que aumenta el número de cartas a enviar, n, la probabilidad de que se produzca un desbarajuste se aproxima a $e^{-1}$. Finalmente obtenemos:
\end{justify}

\begin{align*}
  \lim_{n\to\infty} P(D)&=\lim_{n\to\infty} D(n)=e^{-1}=\frac{1}{e}\\
  \lim_{n\to\infty} P(A)&=1-\frac{1}{e} 
\end{align*}

\end{document}
