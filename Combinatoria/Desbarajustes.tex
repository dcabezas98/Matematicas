\documentclass[12pt,spanish]{article}

% aprovechamiento de la p\'agina -- fill an A4 (210mm x 297mm) page
% Note: 1 inch = 25.4 mm = 72.27 pt
% 1 pt = 3.5 mm (approx)

% vertical page layout -- one inch margin top and bottom
\topmargin      -10 mm   % top margin less 1 inch
\headheight       0 mm   % height of box containing the head
\headsep          0 mm   % space between the head and the body of the page
\textheight     255 mm   % the height of text on the page
\footskip         7 mm   % distance from bottom of body to bottom of foot

% horizontal page layout -- one inch margin each side
\oddsidemargin    0 mm     % inner margin less one inch on odd pages
\evensidemargin   0 mm     % inner margin less one inch on even pages
\textwidth      159 mm     % normal width of text on page

\usepackage[doument]{ragged2e}
\usepackage{babel}
\usepackage[utf8]{inputenc}
\usepackage{amsmath,amsthm,mathtools}
\usepackage{amsfonts,amssymb,latexsym}
\usepackage{enumerate}
\usepackage[dvips,usenames]{color}
\definecolor{RojoAnayelRey}{rgb}{1,.25,.25}
\usepackage{tikz}
\usepackage[bookmarks=true,
            bookmarksnumbered=false, % true means bookmarks in 
                                     % left window are numbered                         
            bookmarksopen=false,     % true means only level 1
                                     % are displayed.
            colorlinks=true,
            linkcolor=webred]{hyperref}
\definecolor{webgreen}{rgb}{0, 0.5, 0} % less intense green
\definecolor{webblue}{rgb}{0, 0, 0.5}  % less intense blue
\definecolor{webred}{rgb}{0.5, 0, 0}   % less intense red
\definecolor{dkgreen}{rgb}{0,0.6,0}
\definecolor{gray}{rgb}{0.5,0.5,0.5}
\definecolor{mauve}{rgb}{0.58,0,0.82}
\definecolor{MistyRose}{RGB}{255,228,225}
\definecolor{LightCyan}{RGB}{224,255,255}

\usepackage{beton}
\usepackage[T1]{fontenc}

% Theorem environments

%% \theoremstyle{plain} %% This is the default
\newtheorem{theorem}{Teorema}[section]
\newtheorem{corollary}[theorem]{Corolario}
\newtheorem{lemma}[theorem]{Lema}
\newtheorem{proposition}[theorem]{Proposici\'on}
%\newtheorem{ax}{Axioma}

\theoremstyle{definition}
\newtheorem{definition}{Definici\'on}[section]
\newtheorem{algorithm}{\textrm{\bf Algoritmo}}[section]

\theoremstyle{remark}
\newtheorem{remark}{Observaci\'on}[section]
\newtheorem{example}{Ejemplo}[section]
\newtheorem{exercise}{Ejercicio}[section]
%\newenvironment{solution}{\begin{proof}[Solution]}{\end{proof}}
\newenvironment{solution}{\begin{proof}[Solución]}{\end{proof}}
\newtheorem*{notation}{Notaci\'on}

%\numberwithin{equation}{section}

%\newcommand{\regla}[2]{
%\begin{array}{c}
%#1\\
%\hline
%#2\\
%\end{array}
%}

\title{Desbarajustes}
\date{}

\begin{document}

\maketitle

\begin{justify}
  Los desbarajustes de n elementos son las formas de ordenar dichos
  elementos de forma que ninguno de ellos acabe en su posición
  correspondiente. Por ejemplo, la permutación $\{3,2,1,5,4\}$ no es
  un desbarajuste porque el 2 está en su posición, mientras que
  $\{5,4,2,1,3\}$ sí que lo es.  El número de desbarajustes para n
  elementos viene dado por la siguiente fórmula:
\end{justify}

\begin{align*}
  D(n)=n!\cdot\sum\limits_{i=0}^n\frac{(-1)^i}{i!}
\end{align*}

\begin{justify}
  Veamos como se deduce la fórmula, para ello primero consideraremos
  el complementario, sea $A\subset\mathnormal{Perm}(n)$ el subconjunto
  de las permutaciones de n elementos que tienen algún elemento en su
  posición correspondiente. Notaremos $A_i$ a las permutaciones que
  conservan la posición del elemento i-ésimo, entonces tenemos:
  
  $A=A_1\cup A_2\cup\ldots\cup A_r \cup\ldots\cup A_n$,
  pero los $A_i$
  no son disjuntos, por ejemplo la permutación que conserva las
  posiciones de los dos primeros elementos, pertenece tanto a $A_1$
  como $A_2$. Luego para hallar el cardinal de $A$ haremos uso del
  principio de inclusión-exclusión ( $|\cdot|$ denota cardinal).
\end{justify}

\begin{justify}
  Cada $A_i$ fija el elemento i-ésimo y permite las permutaciones de
  los $n-1$ elementos restantes, $|A_i|=(n-1)!$, y tenemos n conjuntos
  como este distintos, que podemos escribir como $\binom{n}{1}$.
\end{justify}

\begin{justify}
  Cada $A_i\cap A_j$ fija los elementos i-ésimo y j-ésimo y permite
  las permutacionesde los $n-2$ elementos restantes, $|A_i\cap
  A_j|=(n-2)!$, y podemos formar $\binom{n}{2}$ conjuntos como este
  distintos ($i<j$, i debe ser distinto de j y el orden no importa ya
  que $A_i\cap A_j =A_j\cap A_i$).
\end{justify}

\begin{justify}
  Para cada r tal que $1\leq r \leq n$, tenemos $|A_{i1}\cap
  A_{i2}\cap...\cap A_{ir}|=(n-r)!$, ya que fija r elementos y permite
  las permutaciones de los n-r restantes. Y podemos formar
  $\binom{n}{r}$ conjuntos como este diferentes.
\end{justify}

\begin{justify}
  Finalmente, tenemos $|\bigcap\limits_{i=1}^n A_i|=(n-n)!=1$, y sólo
  existe un conjunto de este tipo,$\binom{n}{n}$.
\end{justify}

\begin{align*}
  |A|&=|A_1\cup A_2\cup\cdots\cup A_r \cup\cdots\cup A_n|\\
  &=\sum\limits_{i=1}^n|A_i|-\sum\limits_{i<j}^n|A_i\cap A_j|+\cdots+(-1)^{n+1}|\bigcap\limits_{i=1}^nA_i|\\
  &=\binom{n}{1}(n-1)!-\binom{n}{2}(n-2)!+\cdots+(-1)^{n+1}\binom{n}{n}(n-n)!\\
  &=\sum\limits_{i=1}^n(-1)^{i+1}\binom{n}{i}(n-i)!
\end{align*}

\begin{justify}
  Pero lo que nos interesan son las permutaciones en las que ningún
  elemento está en su posición correspondiente, es decir, todas las
  permutaciones de n elementos que no pertenezcan a A. Por tanto los
  desbarajustes de n elementos son:
\end{justify}

\begin{align*}
  D(n)&=n!-\sum\limits_{i=1}^n(-1)^{i+1}\binom{n}{i}(n-i)!\\
  &=n!+\sum\limits_{i=1}^n(-1)^i\binom{n}{i}(n-i)!\\
  &=\sum\limits_{i=0}^n(-1)^i\binom{n}{i}(n-i)!
\end{align*}

\begin{justify}
  En la primera igualdad, hemos cambiado el signo a cada elemento de
  la sumatoria, por lo que $(-1)^{i+1}$ que pasa a ser $(-1)^i$, y en
  la segunda, hay que notar que n! es justo el término 0 de la
  sumatoria. El último paso es simplificar el número combinatorio y
  sacar n! de la sumatoria para obtener la expresión del principio.
\end{justify}

\begin{align*}
  D(n)&=\sum\limits_{i=0}^n(-1)^i\binom{n}{i}(n-i)!\\
  &=\sum\limits_{i=0}^n(-1)^i\frac{n!}{i!(n-i)!}(n-i)!\\
  &=n!\sum\limits_{i=0}^n\frac{(-1)^i}{i!}
\end{align*}

\end{document}
