\documentclass{article}
\usepackage[left=3cm,right=3cm,top=2cm,bottom=2cm]{geometry} % page settings
\usepackage{amsmath,amsthm,mathtools}
\usepackage{amsfonts,amssymb,latexsym}
\usepackage{upgreek}

\usepackage[spanish]{babel}
\usepackage[doument]{ragged2e}

\selectlanguage{spanish}
\usepackage[utf8]{inputenc}
\setlength{\parindent}{0mm}

\newcommand{\closure}[1]{\mkern
  1.5mu\overline{\mkern-1.5mu#1\mkern-1.5mu}\mkern 1.5mu}

\begin{document}

\title{Sucesiones en Topología}
\author{David Cabezas Berrido}
\date{}
\maketitle

\section*{Introducción}

\begin{justify} Las sucesiones son herramientas muy usadas en el
  análisis matemático, tanto en la formulación y demostración de
  resultados como en la resolución de problemas concretos. Casi todas
  las propiedades analíticas y topológicas tienen una caracterización
  a través de sucesiones, como la continuidad de una función o la
  adherencia de un conjunto.
\end{justify}

\begin{justify} Sin embargo, su uso en espacios topológicos generales
  es menos frecuente. A continuación veremos algunos ejemplos de
  comportamientos indeseables de sucesiones en espacios topológicos
  por los cuales, a diferencia de los métricos, preferimos evitarlas.
  Así como las propiedades que deben cumplir los espacios para que las
  sucesiones sean una herramienta útil.
\end{justify}

\section*{Sucesiones en espacios topológicos}

\subsection*{Definición: Sucesión}

\begin{justify} Una \textbf{sucesión} es una aplicación que tiene como
  dominio el conjunto de los naturales y cuyo codominio es un conjunto
  cualquiera, en este caso trabajaremos sobre un espacio topológico
  $(X, \mathcal{T})$. \vspace{-4mm}
\end{justify}

\[a : \mathbb{N} \longrightarrow (X, \mathcal{T})\] \\ \vspace{-9mm}
\[n \longrightarrow a_n\]

\begin{justify} La aplicación $a$ asigna a cada número natural un
  elemento $a_n$ cualquiera de $X$, a estos elementos se les llama
  términos de la sucesión.
\end{justify}

\begin{justify} Una sucesión $\{a_n\}_{n \in \mathbb{N}}$ siempre
  tiene infinitos términos (uno por cada elemento de $\mathbb{N}$),
  pero el conjunto de términos de la sucesión
  $\{a_n \text{ } | \text{ } n \in \mathbb{N}\}$ no tiene porqué ser
  infinito, ya que los términos pueden repetirse.
\end{justify}

\subsection*{Definición: Convergencia}

\begin{justify} Decimos que una sucesión $\{a_n\}_{n \in \mathbb{N}}$
  es \textbf{convergente} a un punto $x \in X$ cuando todo entorno de
  $x$ contiene todos los términos de la sucesión a partir de un
  $m \in \mathbb{N}$. Es decir: \vspace{-1mm}

  \[\forall U \in \mathcal{U}^x, \hspace{1mm} \exists m \in \mathbb{N}
    \text{ cumpliendo } a_n \in U \hspace{2mm} \forall n \geq m \]
\end{justify}

\begin{justify} Si $\{a_n\}_{n \in \mathbb{N}}$ converge a $x$,
  decimos que $x$ es el límite de la sucesión lo notaremos $\{a_n\}_{n
    \in \mathbb{N}} \longrightarrow x$
\end{justify}

\begin{justify} En un espacio métrico $(E, d)$, podemos simplificar
  esta definición refiriéndonos a los entornos básicos. Tomamos $\beta^x
  = \{B(x,r) \text{ } | \text{ } r \in \mathbb{R}^+_0\}$ como base de
  entornos de un punto $x \in E$, la definición quedería de la siguiente
  forma:

  \[\{a_n\}_{n \in \mathbb{N}} \longrightarrow x \iff \forall
    \upvarepsilon > 0 \text{ } \exists m \in \mathbb{N} \text{ cumpliendo
    } a_n \in B(x, \upvarepsilon) \hspace{2mm} \forall n \geq m\]
\end{justify}

\begin{justify} En un espacio métrico tenemos entornos tan pequeños
  como queramos, por lo que el hecho de que la sucesión converja nos
  permite ``acercarnos'' al punto $x$ tanto como queramos. Pero en
  general, esto no es posible, ya que existen espacios topológicos en
  los que los entornos están más limitados y permiten que la sucesión
  converja sin ``acercarse'' al límite. Además, en un espacio métrico,
  el límite de una sucesión convergente es único, propiedad que
  tampoco podemos asegurar en un espacio topológico general.
\end{justify}

\section*{Unicidad del límite}

\begin{justify} Ilustremos con un ejemplo como una sucesión puede
  tener, no sólo más de un límite, si no que converja a todos los puntos
  del espacio a la vez.
\end{justify}

\begin{justify} Sea $x_0 \in X$ un punto cualquiera, definimos la
  topología $\mathcal{T}_{x_0}$ de la siguiente forma:
  \[\mathcal{T}_{x_0} = \{\mathcal{O} \subseteq X \text{ }|\text{ }
    x_0 \in \mathcal{O}\} \cup \emptyset\]
\end{justify}

\begin{justify} No es difícil comprobar que esta familia determina una
  topología en $X$. Se conoce como la topología del punto incluido.
\end{justify}

\begin{justify} Tomemos en $(X, \mathcal{T}_{x_0})$ la sucesión
  $\{x_n\}_{n \in \mathbb{N}}$ como la constante $x_0$. Es decir, $x_n =
  x_0 \hspace{2mm} \forall n \in \mathbb{N}$. Probemos $\{x_n\}_{n \in
    \mathbb{N}} \longrightarrow y \hspace{2mm} \forall y \in X$.
\end{justify}

\begin{justify} Sea $y \in X$ arbitrario, todo entorno $U$ de $y$ debe
  contener un abierto $\mathcal{O} \in \mathcal{T}_{x_0}$ cumpliendo $y
  \in \mathcal{O} \subseteq U$. Como $\mathcal{O}$ es abierto, entonces
  $x_0 \in \mathcal{O}$ y por tanto $x_n \in \mathcal{O} \subseteq U
  \hspace{2mm} \forall n \in \mathbb{N}$, luego $\{x_n\}_{n \in
    \mathbb{N}} \longrightarrow y$ como queríamos.
\end{justify}

\begin{justify} Ante semejante problema, es natural preguntarse qué
  propiedades tiene que cumplir un espacio para que toda sucesión
  converjente de elementos del mismo tenga un único límite.
\end{justify}

\subsection*{Definición: Espacio de Hausdorff}

\begin{justify} Se dice que un espacio topológico $(X, \mathcal{T})$
  es un \textbf{espacio de Hausdorff} (o $T_2$) si satisface la
  \textbf{propiedad de Hausdorff} (también llamada \textbf{axioma de
  separación $T_2$}). Esto es, para cada par de puntos distintos de
  $X$, existen entornos disjuntos de cada uno.

  \[\forall x, y \in X \text{ cumpliendo } x \neq y, \hspace{1mm}
    \exists U \in \mathcal{U}^x, V \in \mathcal{U}^y \text{ con } U
    \cap V = \emptyset\]
\end{justify}

\begin{justify} Cuando Felix Hausdorff enunció la definición de
  espacio topológico en 1914, consideró esta propiedad como un cuarto
  axioma que todo espacio topológico debía cumplir. Aunque
  posteriormente se modificó la definición debido a que algunos
  espacios que no poseían esta propiedad también eran objeto de
  interés entre los topólogos.
\end{justify}

\begin{justify} A continuación probaremos que \textbf{en un espacio de
    Hausdorff, el límite de una sucesión convergente es único.}
\end{justify}

\begin{justify} Consideremos en el espacio de Hausdorff
  $(X, \mathcal{T})$, la sucesión $\{a_n\}_{n \in \mathbb{N}}$
  cumpliendo $\{a_n\}_{n \in \mathbb{N}} \longrightarrow x \in X$ y
  $\{a_n\}_{n \in \mathbb{N}} \longrightarrow y \in X$. Llegaremos a
  un absurdo suponiendo $x \neq y$.
\end{justify}

\begin{justify} Como $(X, \mathcal{T})$ es Hausdorff, $\exists U \in
  \mathcal{U}^x, V \in \mathcal{U}^y$ con $U \cap V = \emptyset$. Como
  la sucesión converge a $x$ se tiene
  \[\exists m_x \in \mathbb{N} \text{ cumpliendo } a_n \in U
    \hspace{2mm} \forall n \geq m_x\] Del mismo modo, como la sucesión
  también converge a $y$ obtenemos
  \[\exists m_y \in \mathbb{N} \text{ cumpliendo } a_n \in V
    \hspace{2mm} \forall n \geq m_y\] Tomamos
  $M = \text{max}\{m_x, m_y\} \in \mathbb{N}$, entonces
  $a_M \in U \cap V$, llegando así a una contradicción ya que
  $U \cap V = \emptyset$.
\end{justify}

\section*{Caracterización de la adherencia de un conjunto}

\subsection*{Definición: Adherencia}

\begin{justify} Sea $(X, \mathcal{T})$ un espacio topológico,
  consideramos un subconjunto $A \subseteq X$. Se llama adherencia o
  cierre de $A$ al conjunto
  \[\{x \in X \text{ }|\text{ } \forall U \in
    \mathcal{U}^x,\hspace{1mm} U \cap A \neq \emptyset\}\]
  y lo notamos $\closure{A}$.
\end{justify}

\end{document}