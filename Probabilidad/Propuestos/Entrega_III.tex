\documentclass[tikz]{article}
\usepackage[left=1.8cm,right=3cm,top=1.5cm,bottom=2cm]{geometry} % page
% settings
\usepackage{multicol} 
\usepackage{amsmath} % provides many mathematical environments & tools

\DeclareMathOperator{\Var}{Var}
\DeclareMathOperator{\Cov}{Cov}

\usepackage{dsfont}
\usepackage{upgreek}

\usepackage{scalerel}
\def\stretchint#1{\vcenter{\hbox{\stretchto[440]{\displaystyle\int}{#1}}}}

\usepackage[english]{babel}
\usepackage[doument]{ragged2e}

\usepackage{tikz}
\tikzset{
  every point/.style = {radius={\pgflinewidth}, opacity=1, draw, solid, fill=white},
  pt/.pic = {
    \begin{pgfonlayer}{foreground}
      \path[every point, #1] circle;
    \end{pgfonlayer}
  },
  point/.style={insert path={pic{pt={#1}}}}, point/.default={},
  colored point/.style = {point={fill=#1}},
  point name/.style = {insert path={coordinate (#1)}}
}

% Images
\usepackage{graphicx}
\usepackage{float}
\usepackage{subfigure} % subfiguras
\usepackage{caption}
\captionsetup[table]{labelformat=empty}
\captionsetup[figure]{labelformat=empty}

\usepackage{listings}
\usepackage{xcolor}
\definecolor{gray}{rgb}{0.5,0.5,0.5}
\newcommand{\n}[1]{{\color{gray}#1}}
\lstset{numbers=left,numberstyle=\small\color{gray}}

\selectlanguage{english}
\usepackage[utf8]{inputenc}
\setlength{\parindent}{0mm}

\begin{document}

\title{Ejercicios propuestos: Parte III}
\author{David Cabezas Berrido}
\date{}
\maketitle

\section*{Ejercicio 1}

Función masa de probabilidad:

\begin{center}
\begin{tabular}{|c|c c|}
\hline
\multicolumn{1}{|c|}{$X / Y$}& 0& 2\\\hline
  2      & 1/16 & 1/16 \\
  4      & 1/8 & 1/4 \\
  6 & 0 & 1/2 \\ \hline
    
\end{tabular}
\end{center}

\vspace{3mm}

Se pide calcular la función generatriz de momentos y, a partir de
ella, la matriz de covarianzas.

\vspace{5mm}

\textbf{f.g.m.:}
\[M(t_1,t_2) = E[e^{t_1x+t_2y}] = \frac{1}{16}e^{2t_1} +
  \frac{1}{16}e^{2t_1+2t_2} + \frac{1}{8}e^{4t_1} +
  \frac{1}{4}e^{4t_1+2t_2} + \frac{1}{2}e^{6t_1+2t_2} =
  \frac{1}{16}(e^{2t_1} + e^{2t_1+2t_2} + 2e^{4t_1} + 4e^{4t_1+2t_2}
  + 8e^{6t_1+2t_2})\]

\vspace{2mm}

Queremos calcular la \textbf{matriz de covarianzas}

\begin{center}
  $
  \begin{pmatrix}
    \Var(X) & \Cov(X,Y) \\
    \Cov(X,Y) & \Var(Y) 
  \end{pmatrix}
  =
  \begin{pmatrix}
    \mu_{20} & \mu_{11} \\
    \mu_{11} & \mu_{02} 
  \end{pmatrix}
  $
\end{center}

\vspace{2mm}

Para hallar los momentos centrados, podemos hallar primero los no
centrados. Usaremos la función generatriz de momentos para hallarlos.

\[m_{10} = \frac{\partial M}{\partial t_1}(t_1,t_2)\Big|_{t_1=t_2=0} =
  \frac{1}{16}(2e^{2t_1} + 2e^{2t_1+2t_2} + 8e^{4t_1} +
  16e^{4t_1+2t_2} + 48e^{6t_1+2t_2})\Big|_{t_1=t_2=0} =
  \frac{2+2+8+16+48}{16} = \frac{19}{4}\]

\[m_{01} = \frac{\partial M}{\partial t_2}(t_1,t_2)\Big|_{t_1=t_2=0} =
  \frac{1}{16}(2e^{2t_1+2t_2} + 8e^{4t_1+2t_2} +
  16e^{6t_1+2t_2})\Big|_{t_1=t_2=0} = \frac{2+8+16}{16} = \frac{13}{8}\]

\[m_{20} = \frac{\partial^2 M}{\partial
    t_1^2}(t_1,t_2)\Big|_{t_1=t_2=0} = \frac{1}{16}(4e^{2t_1} +
  4e^{2t_1+2t_2} + 32e^{4t_1} + 64e^{4t_1+2t_2} +
  288e^{6t_1+2t_2})\Big|_{t_1=t_2=0} = \frac{4+4+32+64+288}{16} =
  \frac{49}{2}\]

\[m_{02} = \frac{\partial^2 M}{\partial
    t_2^2}(t_1,t_2)\Big|_{t_1=t_2=0} = \frac{1}{16}(4e^{2t_1+2t_2} +
  16e^{4t_1+2t_2} + 32e^{6t_1+2t_2})\Big|_{t_1=t_2=0} =
  \frac{4+16+32}{16} = \frac{13}{4}\]

\[m_{11} = \frac{\partial^2 M}{\partial
    t_1t_2}(t_1,t_2)\Big|_{t_1=t_2=0} = \frac{1}{16}(4e^{2t_1+2t_2} +
  32e^{4t_1+2t_2} + 96e^{6t_1+2t_2})\Big|_{t_1=t_2=0} =
  \frac{4+32+96}{16} = \frac{33}{4}\]

\vspace{3mm}

Ya podemos hallar los momentos centrados:
\begin{align*}
  \mu_{20}&=m_{20}-m_{10}^2 = \frac{49}{2} - \Big(\frac{19}{4}\Big)^2 = \frac{31}{16} \\
  \mu_{11}&=m_{11}-m_{10}m_{01} = \frac{33}{4} - \frac{19}{4}\frac{13}{8} = \frac{17}{32} \\
  \mu_{02}&=m_{02}-m_{01}^2 = \frac{13}{4} - \Big(\frac{13}{8}\Big)^2 = \frac{39}{64}
\end{align*}

Por tanto la matriz de covarianzas es

\begin{center}
  $
  \begin{pmatrix}
    \frac{31}{16} & \frac{17}{32} \vspace{2mm}\\
    \frac{17}{32} & \frac{39}{64} 
  \end{pmatrix}
  $
\end{center}

\end{document}