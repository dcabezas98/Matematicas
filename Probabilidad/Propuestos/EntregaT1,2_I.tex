\documentclass{article}
\usepackage[left=1.8cm,right=3cm,top=1.5cm,bottom=2cm]{geometry} % page
% settings
\usepackage{multicol} 
\usepackage{amsmath} % provides many mathematical environments & tools
\usepackage{dsfont}
\usepackage{upgreek}
\usepackage[spanish]{babel}
\usepackage[doument]{ragged2e}

% Images
\usepackage{graphicx}
\usepackage{float}
\usepackage{subfigure} % subfiguras
\usepackage{caption}
\captionsetup[table]{labelformat=empty}
\captionsetup[figure]{labelformat=empty}

\usepackage{listings}
\usepackage{xcolor}
\definecolor{gray}{rgb}{0.5,0.5,0.5}
\newcommand{\n}[1]{{\color{gray}#1}}
\lstset{numbers=left,numberstyle=\small\color{gray}}

\selectlanguage{spanish}
\usepackage[utf8]{inputenc}
\setlength{\parindent}{0mm}

\begin{document}

\title{Ejercicios propuestos: Temas 1 y 2 (Parte I)}
\author{David Cabezas Berrido}
\date{}
\maketitle

\section*{Ejercicio 1}
Demostrar la propiedad de falta de memoria de la distribución exponencial:
\[P(X>t+h \ | \ X>h) = P(X>t), \hspace{3mm} t,h > 0, \hspace{3mm} X \sim exp(\lambda)\]

\begin{align*}
  P(X>t+h \ | \ X>h) &= \frac{P(X>t+h \ \cap \ X>h)}{P(X>h)} =^{(*)}
\frac{P(X>t+h)}{P(X>h)} = \frac{1-P(X \leq t+h)}{1-P(X \leq h)} \\
&=\frac{1-F_X(t+h)}{1-F_X(h)} =
\frac{1-(1-e^{-\lambda(t+h)})}{1-(1-e^{-\lambda h})} =
\frac{e^{-\lambda(t+h)}}{e^{-\lambda h}} = \frac{e^{-\lambda t}e^{-\lambda h}}{e^{-\lambda h}} \\ &= e^{-\lambda t} = 1-(1-e^{-\lambda t}) =
1-F_X(t) = 1-P(X \leq t) = P(X>t)
\end{align*}

$(*):$ $X>t+h \subset X>h$.

\section*{Ejercicio 2}
Definir un experimento aleatorio cualquiera, especificando los
elementos del espacio probabilístico $(\Omega, \mathcal{A}, P)$, y
describir adecuadamente sobre él un vector aleatorio discreto
$X = (X_1, X_2 , \ldots, X_n)$ (con n siendo una cifra a tu elección)
que mida una serie de características del experimento. Asimismo,
calcular la función de distribución de dicho vector aleatorio.

\vspace{10mm}

El caballero Ramón decide apostar el dinero de la misión en la ruleta.
Si tiene suerte ganará una fortuna, lo suficiente para comprar su
propia nave estacial. Si no, tendrá que conformarse con invitar a una
damisela a cenar.

Tenemos una ruleta con \textbf{una casilla verde (0), 6 casillas rojas
  (1-6) y 6 casillas negras (1-6)}. De las 6 casillas de un mismo
color, la mitad son pares y la otra mitad impares.

\[\Omega = \{0V, 1R, 2R, 3R, 4R, 5R, 6R, 1N, 2N, 3N, 4N, 5N, 6N\}, \hspace{4mm} \mathcal{A} = \mathcal{P}(\Omega)\]

Supondremos la ruleta no cargada, luego $P(\omega)=\frac{1}{13}$ $\forall \omega \in \Omega$.

Definimos el vector aleatorio $(X,Y)$ donde $X$ mide la paridad del número obtenido e $Y$ el color. \\

$X(\omega)=
\begin{cases}
  0 \text{ si } \omega \in \{0V,2R,4R,6R,2N,4N,6N\} \\
  1 \text{ si } \omega \in \{1R,3R,5R,1N,3N,5N\}
\end{cases}
$

\vspace{4mm}

$Y(\omega)=
\begin{cases}
  0 \text{ si } \omega \in \{0V\} \\
  1 \text{ si } \omega \in \{1R,2R,3R,4R,5R,6R\} \\
  2 \text{ si } \omega \in \{1N,2N,3N,4N,5N,6N\}
\end{cases}
$

\newpage

Función masa de probabilidad: 
\begin{itemize}
\item $P(0,0)=P(\omega \in \{0V\}) = \frac{1}{13}$
\item $P(0,1)=P(\omega \in \{2R,4R,6R\}) = \frac{3}{13}$
\item $P(0,2)=P(\omega \in \{2N,4N,6N\}) = \frac{3}{13}$
\item $P(1,1)=P(\omega \in \{1R,3R,5R\}) = \frac{3}{13}$
\item $P(1,2)=P(\omega \in \{1N,3N,5N\}) = \frac{3}{13}$
\item $P(X,Y)=0$ en el resto de casos
\end{itemize}

Función de distribución:
$F(x,y)=
\begin{cases}
  \hspace{1mm }0 \text{\hspace{1.2mm}si $x < 0$ ó $y < 0$} \\
  \frac{1}{13} \text{ si $0\leq x$, $0\leq y < 1$} \\
  \frac{4}{13} \text{ si $0\leq x < 1$, $1\leq y < 2$} \\
  \frac{7}{13} \text{ si $0\leq x < 1$, $2\leq y$} \\
  \frac{7}{13} \text{ si $1\leq x$, $1\leq y < 2$} \\
  \frac{13}{13} \text{ si $1\leq x$, $2\leq y$}
\end{cases}
$

\section*{Ejercicio 3}

Dar la expresión de las siguientes probabilidades en términos de la
función de distribución para un vector aleatorio $X = (X_1 , X_2)$:
\begin{itemize}
\item $P[a < X_1 < b, c < X_2 < d] = F(b^-,d^-) - F(b^-,c) - F(a,d^-) + F(a,c)$
\item $P[a \leq X_1 < b, c < X_2 < d] = F(b^-,d^-) - F(b^-,c) - F(a^-,d^-) + F(a^-,c)$
\item $P[a < X_1 \leq b, c < X_2 < d] = F(b,d^-) - F(b,c) - F(a,d^-) + F(a,c)$
\item $P[a \leq X_1 \leq b, c < X_2 < d] = F(b,d^-) - F(b,c) - F(a^-,d^-) + F(a^-,c)$
\item $P[a < X_1 < b, c \leq X_2 < d] = F(b^-,d^-) - F(b^-,c^-) - F(a,d^-) + F(a,c^-)$
\item $P[a \leq X_1 < b, c \leq X_2 < d] = F(b^-,d^-) - F(b^-,c^-) - F(a^-,d^-) + F(a^-,c^-)$
\item $P[a < X_1 \leq b, c \leq X_2 < d] = F(b,d^-) - F(b^-,c^-) - F(a,d^-) + F(a,c^-)$
\item $P[a \leq X_1 \leq b, c \leq X_2 < d] = F(b,d^-) - F(b,c^-) - F(a^-,d^-) + F(a^-,c^-)$
\item $P[a < X_1 < b, c < X_2 \leq d] = F(b^-,d) - F(b^-,c) - F(a,d) + F(a,c)$
\item $P[a \leq X_1 < b, c < X_2 \leq d] = F(b^-,d) - F(b^-,c) - F(a^-,d) + F(a^-,c)$
\item $P[a < X_1 \leq b, c < X_2 \leq d] = F(b,d) - F(b,c) - F(a,d) + F(a,c)$
\item $P[a \leq X_1 \leq b, c < X_2 \leq d] = F(b,d) - F(b,c) - F(a^-,d) + F(a^-,c)$
\item $P[a < X_1 < b, c \leq X_2 \leq d] = F(b^-,d) - F(b^-,c^-) - F(a,d) + F(a,c^-)$
\item $P[a \leq X_1 < b, c \leq X_2 \leq d] = F(b^-,d) - F(b^-,c^-) - F(a^-,d) + F(a^-,c^-)$
\item $P[a < X_1 \leq b, c \leq X_2 \leq d] = F(b,d) - F(b,c^-) - F(a,d) + F(a,c^-)$
\item $P[a \leq X_1 \leq b, c \leq X_2 \leq d] = F(b,d) - F(b,c^-) - F(a^-,d) + F(a^-,c^-)$
\end{itemize}

\end{document}