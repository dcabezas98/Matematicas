\documentclass[12pt]{article}

\usepackage[left=2.5cm,right=2cm,top=2cm,bottom=2cm]{geometry}
\setlength{\parindent}{0mm}

\usepackage{float}

\usepackage{parskip}
\usepackage[document]{ragged2e}
\usepackage{babel}
\usepackage[utf8]{inputenc}
\usepackage{amsmath,amsthm,mathtools}
\usepackage{amsfonts,amssymb,latexsym}
\usepackage{enumerate}
\usepackage[dvips,usenames]{color}
\definecolor{RojoAnayelRey}{rgb}{1,.25,.25}
\usepackage{tikz}
\usepackage[bookmarks=true,
            bookmarksnumbered=false, % true means bookmarks in 
                                     % left window are numbered                         
            bookmarksopen=false,     % true means only level 1
                                     % are displayed.
            colorlinks=true,
            urlcolor=cyan,
            linkcolor=blue]{hyperref}
            
\usepackage[T1]{fontenc}

\title{Problema: Cálculo de la esperanza de una variable aleatoria discreta}

\author{David Cabezas Berrido}

\date{}

\begin{document}
\maketitle

La actividad propuesta es la realización de un ejercicio para el alumnado de la asignatura Matemáticas II de 2º de Bachillerato que supone una aplicación del cálculo de la esperanza de una variable aleatoria discreta a una cuestión de la vida cotidiana. La esperanza de una variable aleatoria discreta es un contenido recogido en el Real Decreto 1105/2014, en el bloque 5 (Estadística y Probabilidad) de la asignatura Matemáticas II de 2º de Bachillerato.

Los objetivos de la actividad son:
\begin{itemize}
	\item La práctica del cálculo de la esperanza de una variable aleatoria discreta.
	\item Practicar la identificación de los datos del problema en una situación real.
	\item Despertar interés en los estudiantes sobre la utilidad de éste conocimiento.
\end{itemize}

El ejercicio propuesto es el siguiente:

Ana está haciendo un examen tipo test con 50 preguntas y 3 opciones por pregunta. Cada pregunta acertada suma 1 punto y cada fallo resta 0.5 puntos, de forma que la máxima puntuación posible es de 50 puntos. Tras mucho pensar, Ana está segura de haber respondido correctamente 19 preguntas de las 50; en 12 preguntas de las 31 restantes, ha descartado con certeza una de de las opciones y está dudando entre las otras dos; y en las otras 19 preguntas, Ana no logra sacar ninguna conclusión que le permita descartar alguna opción. En caso de que Ana esté en lo cierto en todas sus suposiciones y responda todas las preguntas cuya respuesta no conoce eligiendo una opción al azar entre las que ella cree posibles, ¿cuál es la nota esperada del examen (sobre 50)?

Resolución:

La dificultad del problema radica en que la distribución de probabilidad de la variable aleatoria ``Nota del examen si en cada pregunta se elige al azar entre las opciones no descartadas'' es bastante incómoda de manejar debido a la cantidad de fenómenos aleatorios que influyen en su valor, ya que tenemos incertidumbre en las respuestas de 31 preguntas. Sin embargo, la puntuación total esperada es la suma de la puntuación esperada en cada una de las preguntas. Puesto que la probabilidad de acertar una pregunta respondiendo al azar sólo depende del número de opciones que queden sin descartar, sólo hay 3 tipos de preguntas en lo que concierne a la puntuación esperada: las que sabe seguro, las que duda entre dos opciones y las que duda entre las tres. Teniendo esto en cuenta, el cálculo de la nota esperada es bastante simple.

Como suponemos que Ana tiene correctas 19 preguntas, la probabilidad de acertar cada una de ellas es 1, luego la puntuación esperada en cada una de ellas es $1\cdot1+0\cdot(-0.5)=1$. En las preguntas en las que Ana duda entre dos opciones, la probabilidad de elegir la correcta es $\frac{1}{2}$, al igual que la probabilidad de fallar. Por tanto, la puntuación esperada en cada una es $\frac{1}{2}\cdot 1+\frac{1}{2}\cdot(-0.5)=0.25$. En el resto de preguntas, la probabilidad de elegir la correcta es $\frac{1}{3}$ y la de fallar $\frac{2}{3}$, por lo que la puntuación esperada en cada una de ellas es $\frac{1}{3}\cdot 1+\frac{2}{3}\cdot (-0.5)=0$.

Finalmente, podemos calcular la nota esperada del examen como la suma de la nota esperada en cada una de las 50 preguntas: $19\cdot 1+12\cdot 0.25+19\cdot 0=22$. Por tanto, la nota esperada será 22 puntos de 50.

El ejercicio se resuelve fácilmente con la definición de la esperanza y aprovechando su aditividad, por lo que el alumnado al que va dirigido cuenta con los conocimientos y herramientas necesarios para su resolución. Además, se espera llamar la atención de los estudiantes aplicando el cálculo de la esperanza para resolver una situación de la vida cotidiana que comúnmente suscita interés entre ellos: calcular la nota esperada en un examen.

El tiempo estimado de la actividad propuesta será de unos 15 o 20 minutos. Se explicará el enunciado en la pizarra o proyector durante los primeros 5 minutos y el alumnado podrá preguntar dudas sobre el enunciado. Después, los estudiantes dispondrán de entre 5 y 10 minutos para la resolución del ejercicio. Finalmente, se procederá a la resolución del ejercicio en pizarra o proyector durante los últimos 5 minutos y se contestarán cuestiones sobre la resolución. Para la realización de la actividad, será necesario el uso de la pizarra o proyector.

\end{document}
