\documentclass[12pt]{article}

\usepackage[left=2.5cm,right=2cm,top=2cm,bottom=2cm]{geometry}
\setlength{\parindent}{0mm}

\usepackage{float}

\usepackage{parskip}
\usepackage[document]{ragged2e}
\usepackage{babel}
\usepackage[utf8]{inputenc}
\usepackage{amsmath,amsthm,mathtools}
\usepackage{amsfonts,amssymb,latexsym}
\usepackage{enumerate}
\usepackage[dvips,usenames]{color}
\definecolor{RojoAnayelRey}{rgb}{1,.25,.25}
\usepackage{tikz}
\usepackage[bookmarks=true,
            bookmarksnumbered=false, % true means bookmarks in 
                                     % left window are numbered                         
            bookmarksopen=false,     % true means only level 1
                                     % are displayed.
            colorlinks=true,
            linkcolor=blue]{hyperref}
            
\usepackage{beton}
\usepackage[T1]{fontenc}

\newcommand{\cov}{\operatorname{cov}}

\title{ACP: Tarea voluntaria}

\author{David Cabezas Berrido}

\date{}

\begin{document}
\maketitle

Justificar que $a_3$ (el vector de pesos de la tercera componente
principal) es el vector de vector propio asociado al tercer valor
propio de mayor módulo de la matriz R.

$a_3$ debe maximizar la expresión $Var[U_3]=Var[a_3^t X]$, siendo un
vector unitario. Y $U_3$ debe ser incorrelada con $U_1$ y con
$U_2$. Esto se traduce en:
\begin{align*}
  \max \ Var[U_3]& \\
  \text{s.a.}\ \|a_3\|=a_3^t a_3=1& \\
  \cov(U_1,U_3)=0& \\
  \cov(U_2,U_3)=0&
\end{align*}

Como $X$ es centrado, tenemos $E[U_3]=E[a_3^t X]=a_3^t E[X]=0$. Por
tanto,
$Var[U_3]=E[U_3^2]=E[a_3^t X a_3^t X]=E[a_3^t XX^ta_3]=a_3^t
E[XX^t]a_3=a_3^t Ra_3$.

Por otra parte, para $i=1,2$ tenemos:
$\cov(U_i,U_3)=E[U_iU_3]-E[U_i]E[U_3]=E[U_i U_3]=E[a_i^t X a_3^t
X]=a_i^t E[X X^t] a_3=a_i^t R a_3$. Llamando $\lambda_i\neq 0$ al
valor propio asociado al vector propio $a_i$ y utilizando que $R$ es
simétrica, tenemos $0=\cov(U_i,U_3)=a_i^t R a_3$=$\lambda_i a_i^t a_3$.
Por tanto el problema queda:
\begin{align*}
  \max\limits_{a_3} & a_3^t Ra_3\\
  \text{s.a.}\ \|a_3\|=a_3^t a_3=1& \\
  a_1^t R a_3=0& \\
  a_2^t R a_3=0& \\
  a_1^t a_3=0& \\
  a_2^t a_3=0&
\end{align*}
Donde las últimas dos condiciones se deducen de las dos anteriores.

Aplicando el Teorema de los multiplicadores Lagrange para la obtención
de extremos condicionados, el problema se reduce a
\[\max\limits_{a_3}\{a_3^t Ra_3 -\lambda(a_3^t a_3-1)-\mu_1 a_1^t R a_3 -\mu_2 a_2^t R a_3\}\]
Derivando la expresión respecto de $a_3$ (matricialmente y teniendo en cuenta que $R$ es simétrica) e igualando a cero, obtenemos:
\begin{equation}\label{derivada}
  2Ra_3-2\lambda a_3-\mu_1 Ra_1-\mu_2 Ra_2=0
  \end{equation}
  Multiplicando a la izquierda por $a_i^t$ ($i=1,2$, llamamos $j$ al
  otro), tenemos:
  \[2a_i^tRa_3-2\lambda a_i^ta_3-\mu_1 a_i^tRa_1-\mu_2 a_i^tRa_2=0\]
  Ahora utilizamos que $a_i^t a_3=a_i^t R a_3=a_i^t R a_j=0$, también
  que $a_i^t Ra_i=\lambda_i\neq 0$. Nos queda $-\mu_i\lambda_i=0$, por
  lo que $\mu_i=0$ para $i=1,2$. Por tanto \ref{derivada} queda:
  \[2Ra_3-2\lambda a_3=0\] Equivalentemente $Ra_3=\lambda a_3$, por lo
  que $a_3$ es un vector propio de $R$ asociado al valor propio
  $\lambda$.

  Además, $Var[U_3]=a_3^t Ra_3=a_3^t \lambda a_3=\lambda$, ya que
  $a_3$ es unitario.

  Por tanto, la tercera componente principal es $U_3=a_3^t X$, siendo
  $a_3$ el vector propio asociado al tercer valor propio de mayor
  módulo de $R$.
\end{document}
