\documentclass[12pt]{article}

\usepackage[left=2.5cm,right=2cm,top=2cm,bottom=2cm]{geometry}
\setlength{\parindent}{0mm}

\usepackage{float}

\usepackage{parskip}
\usepackage[document]{ragged2e}
\usepackage{babel}
\usepackage[utf8]{inputenc}
\usepackage{amsmath,amsthm,mathtools}
\usepackage{amsfonts,amssymb,latexsym}
\usepackage{enumerate}
\usepackage[dvips,usenames]{color}
\definecolor{RojoAnayelRey}{rgb}{1,.25,.25}
\usepackage{tikz}
\usepackage[bookmarks=true,
            bookmarksnumbered=false, % true means bookmarks in 
                                     % left window are numbered                         
            bookmarksopen=false,     % true means only level 1
                                     % are displayed.
            colorlinks=true,
            urlcolor=cyan,
            linkcolor=blue]{hyperref}
            
\usepackage[T1]{fontenc}

\title{Ejercicios sobre la distribución $T^2$ de Hotelling}

\author{David Cabezas Berrido}

\date{}

\begin{document}
\maketitle

\section{Contraste sobre $\mu$ para $\Sigma$ desconocida}

Tenemos los datos de 20 individuos, la Altura (pulgadas) y el Peso
(libras). Suponiendo que siguen una distribución $N_2(\mu,\Sigma)$ con
$\Sigma$ matriz definida no negativa, queremos contrastar las
hipótesis:

\[H_0:\quad \mu=
  \begin{pmatrix}
    70 \\ 170
  \end{pmatrix}=\mu_0;\qquad H_1:\quad \mu\neq \begin{pmatrix}
    70 \\ 170
  \end{pmatrix}=\mu_0
\]

En el caso de que la matriz $\Sigma$ sea desconocida.

Obtenemos los siguientes estadísticos muestrales básicos:

\[\bar{x}=
  \begin{pmatrix}
    71.45 \\ 164.7
  \end{pmatrix} \qquad S_n=
  \begin{pmatrix}
    14.576 & 128.88 \\ 128.88 & 1441.2653
  \end{pmatrix} \qquad r_{12}=0.889\]

Introducimos estos datos en R:
\begin{verbatim}
mu0=matrix(c(70,170), nrow=2, ncol=1) # Valor de mu para la hipótesis nula
x=matrix(c(71.45, 164.7), nrow = 2, ncol = 1) # Vector de medias muestral
# Matriz de cuasi-covarianzas muestral
Sn=matrix(c(14.576,128.88,128.88,1441.2653),nrow=2,ncol=2)
p=2
N=20
n=N-1
r12=0.889 # Coeficiente de correlación muestral
\end{verbatim}

Calculamos el valor del estadístico de contraste:

\begin{verbatim}
> # Estadístico de contraste para Sigma desconocida:
> t=20*t(x-mu0)%*%solve(Sn)%*%(x-mu0)
> t
         [,1]
[1,] 24.65119
\end{verbatim}

\[t=N(\bar{x}-\mu_0)'S_n^{-1}(\bar{x}-\mu_0)=24.65119\]

Valores de comparación teóricos bajo $H_0$ a distintos niveles de significación:

\begin{verbatim}
f01=qf(0.1, 2, 18, lower.tail = FALSE)*38/18 # alpha=0.1
f005=qf(0.05, 2, 18, lower.tail = FALSE)*38/18 # alpha=0.05
f001=qf(0.01, 2, 18, lower.tail = FALSE)*38/18 # alpha=0.01
\end{verbatim}

\begin{verbatim}
> f01
[1] 5.539444
> f005
[1] 7.504065
> f001
[1] 12.69391
\end{verbatim}

\[\frac{38}{18}F_{2,28;0.1}=5.539444 ; \qquad \frac{38}{18}F_{2,28;0.05}=7.504065 ; \qquad \frac{38}{18}F_{2,28;0.01}=12.69391\]

Para los tres niveles de significación se tiene
$24.65119=t>\frac{38}{18}F_{2,28;\alpha}$, por lo que en los tres casos
rechazaríamos la hipótesis nula.

\section{Regiones de confianza en torno a $\mu_0$ para distintos
  valores del nivel de confianza}

\subsection{Caso de $\Sigma$ conocida}

\subsection{Caso de $\Sigma$ desconocida}


\end{document}
