\documentclass{amsart}
\usepackage{amssymb}
\usepackage{amsmath}
\usepackage[latin1]{inputenc}

\usepackage{parskip}
\usepackage{enumerate}

\def\diam{\mathrm{diam}}
\def\dom{\mathrm{dom}}
\def\card{\mathrm{card}}
\def\Re{\mathrm{Re}}
\def\Im{\mathrm{Im}}
\def\lin{\mathrm{lin}}
\def\dim{\mathrm{dim}}
\def\codim{\mathrm{codim}}
\def\co{\mathrm{co}}
\def\Int{\mathrm{Int}}

\setlength{\parindent}{0mm}

\usepackage{vmargin}
\setmargins{2.5cm}       % margen izquierdo
{1.5cm}                        % margen superior
{15.5cm}                      % anchura del texto
{23.42cm}                    % altura del texto
{10pt}                           % altura de los encabezados
{1cm}                           % espacio entre el texto y los encabezados
{0pt}                             % altura del pie de p�gina
{2cm}                           % espacio entre el texto y el pie de p�gina
\begin{document}

\begin{center}
\textbf{Problemas de An\'{a}lisis Funcional Avanzado}

\textbf{Tema 2: Espacios localmente convexos. \\
	Teorema de Krein-Milman}

\smallskip

David Cabezas Berrido

\end{center}

\smallskip

\textbf{Ejercicio 1:} 

\smallskip

Sea $X$ un espacio vectorial sobre $\mathbb{C}$, $X'$ su dual algebraico y $X_\mathbb{R}$, $(X')_\mathbb{R}$ los correspondientes espacios reales subyacentes. Prueba que la aplicaci�n $f\mapsto\Re f$ es una biyecci�n $\mathbb{R}$-lineal de $(X')_\mathbb{R}$ en $(X_\mathbb{R})'$.

\textbf{Soluci\'on:}

Si $f\in (X')_\mathbb{R}$, en particular $f\in X'$ (son id�nticos como conjuntos), luego $f(\lambda x+y)=\lambda f(x)+f(y)$ para cada $\lambda\in\mathbb{C}$ y $x,y\in X$. Podemos escribir $f$ como $f=\Re f+i\Im f$, de modo que $\Re f(\lambda x+y)+i\Im f(\lambda x+y)=\lambda(\Re f(x)+i\Im f(x))+\Re f(y)+i\Im f(y)$ para cada $\lambda\in\mathbb{C}$ y $x,y\in X$. En el caso de que $\lambda$ sea real tendremos $\Re f(\lambda x+y)=\lambda\Re f(x)+\Re f(y)$, lo que demuestra ($\lambda\in\mathbb{R}$ y $x,y\in X$ son arbitrarios) que $\Re f\in (X_\mathbb{R})'$.

Dado $g\in (X_\mathbb{R})'$, tomando $f(x):=g(x)-ig(ix)$ para cada $x\in X$ se tiene
\begin{align*}
f(\lambda x+y)&=g(\lambda x+y)-ig(i\lambda x+iy)=\lambda g(x)+g(y)-i(\lambda g(ix)+g(iy)) \\
&=\lambda g(x)-\lambda ig(ix)+g(y)-ig(iy)=\lambda f(x)+f(y)
\end{align*}
para $x,y\in X$ y $\lambda\in\mathbb{R}$. Esto demuestra que $f$ es $\mathbb{R}$-lineal, pero adem�s
\[f(ix)=g(ix)-ig(-x)=i(-i)g(ix)+ig(x)=i(g(x)-ig(ix))=if(x)\]
para cada $x\in X$. Concluimos que $f$ es ($\mathbb{C}$-)lineal y por tanto $f\in X'=(X')_\mathbb{R}$ (como conjuntos). Como es claro que $\Re f=g$, deducimos que la aplicaci�n es sobreyectiva.

La identidad $\Im f(x)=\Re(-i f(x))=-\Re f(ix)$ sugiere que $f$ viene un�vocamente determinado por su parte real, luego en realidad tenemos una biyecci�n.

Finalmente, es claro que la funci�n parte real es $\mathbb{R}$-lineal.

\smallskip

\textbf{Ejercicio 2:}
\smallskip

Sean $X$ e $Y$ espacios vectoriales y $g:X\rightarrow Y$ una aplicaci�n. Prueba que las siguientes afirmaciones son equivalentes:
\begin{enumerate}[i)]
	\item $g$ es af�n.
	\item Existe una aplicaci�n $\mathbb{R}$-lineal $f:X\rightarrow Y$ y un vector $b\in Y$ tales que $g(x)=f(x)+b$ para todo $x\in X$.
\end{enumerate}

\textbf{Soluci\'on:}

Supongamos primero (ii). Dados $x,y\in X$ y $t\in[0,1]$, tenemos
\begin{align*}
g((1-t)x+ty)&=f((1-t)x+ty)+b=(1-t)f(x)+tf(y)+(1-t)b+tb\\&=(1-t)(f(x)+b)+t(f(y)+b)=(1-t)g(x)+tg(y),
\end{align*}
lo que demuestra que $g$ es af�n.

Suponiendo (i), tomamos $f:X\rightarrow Y$ dada por $f(x)=g(x)-g(0)$. S�lo necesitamos probar que $f$ es $\mathbb{R}$-lineal.
 Primero comprobamos que $f$ preserva el producto por escalares de $[0,1]$, sean $x\in X$ y $t\in[0,1]$:
 \begin{align*}
f(tx)=g(tx)-g(0)=g(tx+(1-t)0)-g(0)=tg(x)+(1-t)g(0)-g(0)=tg(x)-tg(0)=tf(x).
\end{align*}
Ahora vemos que $f$ separa sumas. Si $x,y\in X$, tenemos
\begin{align*}
f(x+y)&=g\Big(\frac{2x+2y}{2}\Big)-g(0)=\frac{1}{2}g(2x)+\frac{1}{2}g(2y)-\frac{1}{2}g(0)-\frac{1}{2}g(0) \\
&=\frac{1}{2}f(2x)+\frac{1}{2}f(2y)=f(x)+f(y),
\end{align*}
donde hemos aplicado lo anterior para cancelar los $\frac{1}{2}$ de fuera con los $2$ de dentro de $f$.

Utilizando la aditividad un n�mero arbitrario de veces podemos concluir que $f(nx)=nf(x)$ para cada $n\in\mathbb{N}\cup\{0\}$ y $x\in X$. Todo real positivo $r\in\mathbb{R}^+$ se puede escribir como $r=n+t$ con $n\in\mathbb{N}\cup\{0\}$ y $t\in[0,1]$. Por tanto,
\[f(rx)=f(nx+tx)=f(nx)+f(tx)=nf(x)+tf(x)=rf(x),\] lo que demuestra que $f$ preserva homotecias de raz�n positiva.
Finalmente, la igualdad
\[0=f(0)=f(x+(-x))=f(x)+f(-x)\]
lleva a $f(-x)=-f(x)$ para todo $x\in X$. Concluimos as� que $f$ es $\mathbb{R}$-lineal.

\smallskip

\textbf{Ejercicio 3:}
\smallskip

Sea $X$ un espacio vectorial, $n$ un natural, $B_j$ con subconjunto convexo de $X$ para todo $j\in\{1,\ldots,n\}$ y $B=\bigcup_{j=1}^n B_j$. Prueba que
\[\co(B)=\left\{\sum_{j=1}^n t_j x_j: t_j\in[0,1], \ x_j\in B_j \ \forall j=1,\ldots,n, \ \sum_{j=1}^n t_j=1\right\}.\]
Observa como consecuencia que si $b_1,\ldots,b_n$ son elementos de $X$,
\[\co(\{b_1,\ldots, b_n\})=\left\{\sum_{j=1}^n t_j b_j: t_j\in[0,1] \ \forall j=1,\ldots,n, \ \sum_{j=1}^n t_j=1\right\}.\]

\textbf{Soluci\'on:}

La segunda igualdad es un caso particular en la que cada $B_j$ se reduce a un punto, por lo que basta con demostrar la primera.
Adem�s, la inclusi�n hacia la derecha ($\supset$) es trivial por la Proposici�n 3.1.1, ya que cada $B_j$ est� contenido en $B$.

Dado $x\in \co(B)$, podemos escribir $x=\sum_{i=1}^m \lambda_i y_i$ con $m\in\mathbb{N}$, $\lambda_i\in[0,1]$, $y_i\in B$ para cada $i=1,\ldots,m$ (Proposici�n 3.1.1) y $\sum_{i=1}^m \lambda_i=1$.
Nuestro objetivo es escribir $x$ con $n$ sumandos cada uno en un conjunto $B_j$ y manteniendo la misma condici�n sobre los pesos (deben ser no negativos y sumar 1).

Supongamos que hubiese dos elementos $y_i,y_{i'}\in B$ con $i,i'\in\{1,\ldots,m\}$, $i\neq i'$, y $y_i,y_{i'}\in B_j$ para alg�n $j=1,\ldots,n$. Reordenando y renombrando si es preciso, podemos suponer que son $y_{m-1},y_m\in B_j$. Por la convexidad de $B_j$ tenemos
\[y:=\frac{\lambda_{m-1}}{\lambda_{m-1}+\lambda_m}y_{m-1}+\frac{\lambda_m}{\lambda_{m-1}+\lambda_m}y_m\in B_j\subset B,\]
y podemos escribir
\[x=\sum_{i=1}^{m-2} \lambda_i y_i+(\lambda_{m-1}+\lambda_m)y.\]
Hemos expresado $x$ utilizando un sumando menos y es obvio que se sigue cumpliendo la condici�n sobre los nuevos puntos y pesos.

Podemos repetir este proceso siempre que existan dos �ndices distintos tales que los puntos correspondientes est�n en un mismo $B_j$, de modo que en un n�mero finito de pasos llegar�amos a expresar $x$ como una combinaci�n convexa en la que cada $y_i$ pertenece a un �nico $B_i$, lo que forzar�a a que esa combinaci�n tuviese como m�ximo $n$ sumandos. En caso de haber menos, podr�amos a�adir nuevos elementos a la combinaci�n otorg�ndoles peso 0, de forma que todos los $B_j$ tengan (como m�nimo) un representante.

\pagebreak

\textbf{Ejercicio 4:}
\smallskip

Sea $X$ un espacio vectorial y $\mathcal{B}$ una familia no vac�a de subconjuntos absorbentes y equilibrados de $X$ de modo que
\begin{enumerate}[i)]
	\item $\forall U,V\in\mathcal{B}, \ \exists W\in \mathcal{B}: W\subset U\cap V$.
	\item $\forall U\in\mathcal{B}, \ \exists V\in \mathcal{B}: V+V\subset U$.
\end{enumerate}
Prueba que existe una �nica topolog�a vectorial en $X$ con respecto a la cual $\mathcal{B}$ es una base de entornos de cero.

\textbf{Soluci\'on:}

\end{document}