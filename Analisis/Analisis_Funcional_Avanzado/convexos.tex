\documentclass{amsart}
\usepackage{amssymb}
\usepackage{amsmath}
\usepackage[latin1]{inputenc}

\usepackage{parskip}

\def\diam{\mathrm{diam}}
\def\dom{\mathrm{dom}}
\def\card{\mathrm{card}}
\def\Re{\mathrm{Re}}
\def\Im{\mathrm{Im}}
\def\lin{\mathrm{lin}}
\def\dim{\mathrm{dim}}
\def\codim{\mathrm{codim}}
\def\co{\mathrm{co}}
\def\Int{\mathrm{Int}}

\setlength{\parindent}{0mm}

\usepackage{vmargin}
\setmargins{2.5cm}       % margen izquierdo
{1.5cm}                        % margen superior
{15.5cm}                      % anchura del texto
{23.42cm}                    % altura del texto
{10pt}                           % altura de los encabezados
{1cm}                           % espacio entre el texto y los encabezados
{0pt}                             % altura del pie de p�gina
{2cm}                           % espacio entre el texto y el pie de p�gina
\begin{document}

\begin{center}
\textbf{Problemas de An\'{a}lisis Funcional Avanzado}

\textbf{Tema 2: Espacios localmente convexos. \\
	Teorema de Krein-Milman}

\smallskip

David Cabezas Berrido

\end{center}

\smallskip

\textbf{Ejercicio 1:} 

\smallskip

Sea $X$ un espacio vectorial sobre $\mathbb{C}$, $X'$ su dual algebraico y $X_\mathbb{R}$, $(X')_\mathbb{R}$ los correspondientes espacios reales subyacentes. Prueba que la aplicaci�n $f\mapsto\Re f$ es una biyecci�n $\mathbb{R}$-lineal de $(X')_\mathbb{R}$ en $(X_\mathbb{R})'$.


\textbf{Soluci\'on:}

Si $f\in (X')_\mathbb{R}$, en particular $f\in X'$ (son id�nticos como conjuntos), luego $f(\lambda x+y)=\lambda f(x)+f(y)$ para cada $\lambda\in\mathbb{C}$ y $x,y\in X$. Podemos escribir $f$ como $f=\Re f+i\Im f$, de modo que $\Re f(\lambda x+y)+i\Im f(\lambda x+y)=\lambda(\Re f(x)+i\Im f(x))+\Re f(y)+i\Im f(y)$ para cada $\lambda\in\mathbb{C}$ y $x,y\in X$. En el caso de que $\lambda$ sea real tendremos $\Re f(\lambda x+y)=\lambda\Re f(x)+\Re f(y)$, lo que demuestra ($\lambda\in\mathbb{R}$ y $x,y\in X$ son arbitrarios) que $\Re f\in (X_\mathbb{R})'$.

Dado $g\in (X_\mathbb{R})'$, tomando $f(x):=g(x)-ig(ix)$ para cada $x\in X$ se tiene
\begin{align*}
f(\lambda x+y)&=g(\lambda x+y)-ig(i\lambda x+iy)=\lambda g(x)+g(y)-i(\lambda g(ix)+g(iy)) \\
&=\lambda g(x)-\lambda ig(ix)+g(y)-ig(iy)=\lambda f(x)+f(y)
\end{align*}
para $x,y\in X$ y $\lambda\in\mathbb{R}$. Esto demuestra que $f$ es $\mathbb{R}$-lineal, pero adem�s
\[f(ix)=g(ix)-ig(-x)=i(-i)g(ix)+ig(x)=i(g(x)-ig(ix))=if(x)\]
para cada $x\in X$. Concluimos que $f$ es ($\mathbb{C}$-)lineal y por tanto $f\in X'=(X')_\mathbb{R}$ (como conjuntos). Como es claro que $\Re f=g$, deducimos que la aplicaci�n es sobreyectiva.

La identidad $\Im f(x)=\Re(-i f(x))=-\Re f(ix)$ sugiere que $f$ viene un�vocamente determinado por su parte real, luego en realidad tenemos una biyecci�n.

Finalmente, es claro que la funci�n parte real es $\mathbb{R}$-lineal.

\smallskip

\textbf{Ejercicio 2:}
\smallskip

\end{document}

