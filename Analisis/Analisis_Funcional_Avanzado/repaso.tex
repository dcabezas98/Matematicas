\documentclass{amsart}
\usepackage{amssymb}
\usepackage{amsmath}
\usepackage[latin1]{inputenc}

\def\N{\mathbb{N}}
\def\Z{\mathbb{Z}}
\def\Q{\mathbb{Q}}
\def\R{\mathbb{R}}
\def\C{\mathbb{C}}
\def\K{\mathbb{K}}

\def\diam{\mathrm{diam}}
\def\dom{\mathrm{dom}}
\def\card{\mathrm{card}}
\def\Re{\mathrm{Re}}
\def\Im{\mathrm{Im}}
\def\lin{\mathrm{lin}}
\def\dim{\mathrm{dim}}
\def\codim{\mathrm{codim}}
\def\co{\mathrm{co}}
\def\Int{\mathrm{Int}}

\usepackage{vmargin}
\setmargins{2.5cm}       % margen izquierdo
{1.5cm}                        % margen superior
{15.5cm}                      % anchura del texto
{23.42cm}                    % altura del texto
{10pt}                           % altura de los encabezados
{1cm}                           % espacio entre el texto y los encabezados
{0pt}                             % altura del pie de p�gina
{2cm}                           % espacio entre el texto y el pie de p�gina
\begin{document}

\begin{center}
\textbf{Problemas de An\'{a}lisis Funcional Avanzado}

\textbf{Tema 1: Principios fundamentales del An\'{a}lisis Funcional (repaso)}

\bigskip

David Cabezas Berrido

\end{center}

\bigskip

\textbf{Ejercicio 1:} Sean $X$ e $Y$ espacios de Banach y $T\colon X\to Y$ una aplicaci\'{o}n lineal y continua. Pruebe que las siguientes afirmaciones son equivalentes:
\begin{enumerate}
	\item $\ker(T)=\{0_X\}$ y $T(X)$ es cerrado en $Y$.
	\item Existe una constante $\alpha>0$ tal que $\alpha \left\|x\right\|\leq\left\|T(x)\right\|$ para todo $x\in X$.
\end{enumerate}

\bigskip

\textbf{Soluci\'on:}

	Supongamos (1) y consideremos $T$ como una aplicaci�n sobreyectiva sobre su imagen, $T:X\rightarrow T(X)$. Adem�s, la condici�n $\ker(T)=\{0_X\}$ nos dice que $T$ es inyectiva, luego tenemos una biyecci�n. Por otra parte, al ser $T(X)$ un subespacio cerrado de un espacio completo $Y$, deducimos que $T(X)$ es tambi�n un espacio de Banach.
	
	Aplicando el teorema de los isomorfismos de Banach concluimos que $T:X\rightarrow T(X)$ es un isomorfismo, por tanto $T^{-1}:T(X)\rightarrow X$ es continua. Esto equivale a que exista $M>0$ tal que
	\[\|T^{-1}(y)\|\leq M\|y\|\quad\forall y\in T(X).\]
	
	Como $T:X\rightarrow T(X)$ es una biyecci�n, decir $y\in T(X)$ es tan arbitrario como decir $T(x)$ con $x\in X$, luego tenemos
	\[\|x\|=\|T^{-1}(T(x))\|\leq M\|T(x)\|\quad\forall x\in X.\]
	Tomando $\alpha=M^{-1}>0$ obtenemos la condici�n deseada, (2).
	
	\smallskip
	
	Para la implicaci�n inversa, la desigualdad en (2) nos dice que si $x\in\ker(T)$, entonces $\alpha\|x\|\leq\|T(x)\|=0$, lo que fuerza (por ser $\alpha>0$) que $x=0$. Concluimos que $\ker(T)=\{0\}$ y que $T:X\rightarrow T(X)$ es una biyecci�n.
	
	Ahora tomamos una sucesi�n convergente cualquiera $\{y_n\}\to y\in Y$ de elementos de $T(X)$, nuestro objetivo es probar que $y\in T(X)$. Sea $\{x_n\}$ la sucesi�n de elementos de $X$ definida por $x_n=T^{-1}(y_n)$, la condici�n (2) nos dice que
	\[\|x_n-x_m\|\leq \alpha^{-1}\|T(x_n-x_m)\|=\alpha^{-1}\|y_n-y_m\| \quad\forall n,m\in\mathbb{N}.\]
	La sucesi�n $\{y_n\}$ es convergente y por tanto de Cauchy, luego la desigualdad anterior nos asegura que $\{x_n\}$ tambi�n es de Cauchy. Como $X$ es completo, existe $x\in X$ tal que $\{x_n\}\to x$. Finalmente, la continuidad de $T$ nos permite deducir que
	\[T(x)=T\left(\lim\limits_{n\to\infty}x_n\right)=\lim\limits_{n\to\infty}y_n=y.\]
	Concluimos que $y\in T(X)$, y de la arbitrariedad de la sucesi�n llegamos a que $T(X)$ es cerrado en $Y$.

\bigskip

\textbf{Ejercicio 2:}
En el contexto de espacios de Banach:
\begin{enumerate}
\item Pruebe el teorema de la aplicaci\'{o}n abierta usando el teorema de los isomorfismos de Banach.
\item Demuestre el teorema de la gr\'{a}fica cerrada usando el teorema de la aplicaci\'{o}n abierta.
\item Pruebe el teorema de los isomorfismos de Banach usando el teorema de la gr\'{a}fica cerrada.
\end{enumerate}

\bigskip

\textbf{Soluci\'on:}

A lo largo del ejercicio, $X$ e $Y$ ser�n dos espacios de Banach y $T:X\rightarrow Y$ una aplicaci�n lineal.

\begin{enumerate}
	\item Supongamos que $T$ es continua y sobreyectiva, debemos probar que es abierta. Como $T$ es continua, $\ker(T)$ es cerrado en $X$. Al ser $X$ completo, el cociente $X/\ker(T)$ es un espacio de Banach. La descomposici�n can�nica nos asegura que la aplicaci�n $\tilde{T}:X/\ker(T)\rightarrow T(X)=Y$ dada por $\tilde{T}(x+\ker(T))=T(x)$ es una biyecci�n lineal continua. Por el teorema de los isomorfismos de Banach $\tilde{T}$ es un isomorfismo, en particular abierta. Como $T$ es la composici�n de $\tilde{T}$ con la proyecci�n cociente (que siempre es abierta), concluimos que $T$ es abierta.
	
	\smallskip
	
	\item Supongamos ahora que $T$ tiene gr�fica cerrada, debemos probar que es continua. Como $\operatorname{Gr}T$ es un subespacio cerrado del espacio de Banach $X\times Y$ (producto de Banach), $\operatorname{Gr}T$ es Banach. Consideramos la aplicaci�n $\Phi:\operatorname{Gr}T\rightarrow X$ dada por $\Phi(x,T(x))=x$, que es continua (es una proyecci�n) y claramente sobreyectiva, luego el teorema de la aplicaci�n abierta nos garantiza que $\Phi$ es abierta. La funci�n $\Phi$ es biyectiva con $\Phi^{-1}(x)=(x,T(x))$, y el hecho de que $\Phi$ sea abierta nos dice que $\Phi^{-1}$ es continua. Concluimos que $T$ es continua por ser una componente de $\Phi^{-1}$.
	
	\smallskip
	
	\item Supongamos que $T$ es continua y biyectiva, debemos probar que es un isomorfismo, es decir, que $T^{-1}:Y\rightarrow X$ es continua. En vista del teorema de la gr�fica cerrada, nos basta con comprobar que $\operatorname{Gr}T^{-1}$ es un subconjunto cerrado de $Y\times X$. Usando que $T$ es una biyecci�n obtenemos
	\[\operatorname{Gr}T^{-1}=\{(y,T^{-1}(y)):y\in Y\}=\{(T(x),x):x\in X\}.\]
	Observamos que $\operatorname{Gr}T^{-1}$ es la imagen de $\operatorname{Gr}T$ por el isomorfismo de $X\times Y$ en $Y\times X$ dado por $(x,y)\mapsto(y,x)$. Como $T$ es continua, $\operatorname{Gr}T$ es cerrada en $X\times Y$, luego $\operatorname{Gr}T^{-1}$ es cerrada en $Y\times X$.
\end{enumerate}

\bigskip

\textbf{Ejercicio 3:} Sea $\left\{y(n)\right\}_{n\in\N}$ una sucesi\'{o}n de n\'umeros reales o complejos tal que la serie $\sum_{n\in\N}x(n)y(n)$ es convergente para toda sucesi\'{o}n $\left\{x(n)\right\}_{n\in\N}\in c_0$. Compruebe que la serie $\sum_{n\in\N}\left|y(n)\right|$ converge.

\bigskip

\textbf{Soluci\'on:}

Comenzamos considerando para cada $n\in\mathbb{N}$ el funcional $f_n:c_0\rightarrow\mathbb{K}$ dado por
\[f_n(x)=\sum_{k=1}^n x(k)y(k),\quad \forall x\in c_0.\] Claramente cada $f_n$ es un funcional lineal, y comprobamos que es continuo con $\|f_n\|=\sum\limits_{k=1}^n|y(k)|$:
para cada $x\in c_0$ tenemos
\[|f_n(x)|\leq\sum_{k=1}^n |x(k)| |y(k)|\leq \sum_{k=1}^n \|x\| |y(k)|=\|x\|\sum\limits_{k=1}^n|y(k)|,\]
pero tomando $x(k)=\frac{\overline{y(k)}}{|y(k)|}$ para $k\leq n$ y $x(k)=0$ para $k>n$ se obtiene la igualdad en la expresi�n anterior.

La hip�tesis de partida nos dice justamente que la familia de funcionales $\{f_n:n\in\mathbb{N}\}\subset c_0^*$ est� puntualmente acotada en $c_0$, y el principio de acotaci�n uniforme para espacios de Banach ($c_0$ es Banach) nos permite concluir que dicha familia est� acotada en norma, luego el conjunto $\{\|f_n\|:n\in\mathbb{N}\}$ est� acotado. Esto quiere decir que las sumas parciales de la serie $\sum_{n\in\N}\left|y(n)\right|$ est�n acotadas, y por monoton�a deducimos que la serie converge.


\bigskip

\textbf{Ejercicio 4:} 
Sean $X$ e $Y$ espacios normados y $x_0\in X\backslash \{0_X\}$. Dado $y_0\in Y$, pruebe que existe $T\in L(X,Y)$ tal que $T(x_0)=y_0$ y $\left\|T\right\| \left\| x_0\right\| =\left\| y_0\right\| $.

\bigskip

\textbf{Soluci\'on:}

Como $x_0\neq 0$, $\mathbb{K}x_0$ es un subespacio de dimensi�n 1 de $X$. Definimos el funcional $g(\lambda x_0)=\lambda$ para cada $\lambda\in\mathbb{K}$, claramente se tiene $g\in(\mathbb{K}x_0)^*$. La igualdad
\[|g(\lambda x_0)|=|\lambda|=\|\lambda x_0\|\frac{1}{\|x_0\|}\]
asegura que $\|g\|=\frac{1}{\|x_0\|}$, y adem�s que la norma se alcanza en cada punto de $\mathbb{K}x_0$. Utilizando el teorema de extensi�n Hahn-Banach para espacios normados encontramos un funcional $f\in X^*$ tal que $f(\lambda x_0)=\lambda$ para cada $\lambda\in\mathbb{K}$ y $\|f\|=\|g\|=\frac{1}{\|x_0\|}$.

Definimos ahora $T:X\rightarrow Y$ por $T(x)=f(x)y_0$, claramente $T\in L(X,Y)$. Tenemos $T(x_0)=f(1\cdot x_0)y_0=1\cdot y_0$, adem�s
\[\|T(x)\|=|f(x)|\|y_0\|\leq \|f\|\|x\|\|y_0\|=\|x\|\frac{\|y_0\|}{\|x_0\|},\]
luego $\|T\|\leq\|f\|\|y_0\|=\frac{\|y_0\|}{\|x_0\|}$. La igualdad se alcanza precisamente en $x_0$ (de hecho en todo $\mathbb{K}x_0$):
\[\|T(x_0)\|=\|y_0\|=\|x_0\|\frac{\|y_0\|}{\|x_0\|}.\]

\bigskip

\textbf{Ejercicio 5:} 
Sea $X$ un espacio normado separable. Pruebe que existe un subconjunto numerable de $X^{*}$ que separa los puntos de $X$.

\bigskip

\textbf{Soluci\'on:}

Sea $E$ un subconjunto denso numerable de $X$. Si $X$ es trivial no hay nada que demostrar, en otro caso para cada $x\in E$ existe (como consecuencia del teorema de Hahn-Banach) $f_x\in X^*$ tal que $\|f_x\|=1$ y $f_x(x)=\|x\|$.

Dados $z,y\in X$ con $z\neq y$, tenemos $z-y\neq 0$. Tomamos $\varepsilon=\|z-y\|>0$, por la densidad existe $x\in E$ tal que $\|x-(z-y)\|<\frac{\varepsilon}{3}$, luego $\|x\|=\|x-(z-y)+(z-y)\|\geq \|z-y\|-\|x-(z-y)\|>\varepsilon-\frac{\varepsilon}{3}=\frac{\varepsilon}{2}$.

Por tanto, $|f_x(z-y)|=|f_x((z-y)-x)+f_x(x)|\geq |f_x(x)|-|f_x((z-y)-x)|\geq\|x\|-\|f_x\|\|(z-y)-x\|>\frac{\varepsilon}{2}-1\cdot\frac{\varepsilon}{3}>0$. Lo cual implica que $f_x(z)\neq f_x(y)$.

Hemos probado que dados dos puntos diferentes de $X$, existe un funcional del conjunto $\{f_x:x\in E\}\subset X^*$ que los separa. La numerabilidad de $E$ nos asegura que $\{f_x:x\in E\}$ es el subconjunto que buscamos.
	
\bigskip

\textbf{Ejercicio 6:} 
Sea $X$ un espacio de Banach, $Y$ un espacio normado y $T\colon X\to Y$ una aplicaci\'{o}n lineal. Se define una nueva norma en $X$ mediante la expresi\'{o}n: 
\[
\Vert x\Vert _1=\Vert x\Vert +\Vert T(x)\Vert ,\quad \forall x\in X.
\]
Pruebe que las siguientes afirmaciones son equivalentes:
\begin{enumerate}
\item  $T$ es continua.
\item  $\Vert \cdot \Vert _1$ es equivalente a $\Vert \cdot \Vert $.
\item  $\Vert \cdot \Vert _1$ es una norma completa.
\end{enumerate}

\bigskip

\textbf{Soluci\'on:}

Probaremos que tanto (3) como (1) son equivalentes a (2).

\smallskip

Si ambas normas son equivalentes, dan lugar a las mismas sucesiones convergentes y a las mismas sucesiones de Cauchy, luego la complitud de una de ellas implica la de la otra. Por tanto, es claro que (2) implica (3).

\smallskip

Supongamos (3). Como las normas $\|\cdot\|_1$ y $\|\cdot\|$ son ambas completas, para ver que son equivalentes basta probar que son comparables en vista de un corolario del teorema de los isomorfismos de Banach. Pero esto es obvio, ya que $\|x\|_1\geq\|x\|$ para cada $x\in X$. Tenemos as� (2).

\smallskip

Ahora supondremos (2) y probaremos (1). Si ambas normas son equivalentes, debe existir una constante $M>0$ tal que
\[\|x\|+\|T(x)\|=\|x\|_1\leq M \|x\|\]
para cada $x\in X$. Tenemos entonces para todo $x\in X$
\[\|T(x)\|\leq (M-1)\|x\|,\]
lo que prueba que $T$ es continua con $\|T\|\leq M-1$.

\smallskip

Falta comprobar que (1) implica (2). Si $T$ es continua, existe una constante $M>0$ tal que $\|T(x)\|\leq M\|x\|$ para todo $x\in X$. Luego
\[\|x\|\leq \|x\|_1=\|x\|+\|T(x)\|\leq \|x\|+M \|x\|=(M+1)\|x\|\]
para cada $x\in X$. Obtenemos as� (2).

\end{document}

