\documentclass{amsart}
\usepackage{amssymb}
\usepackage{amsmath}
\usepackage[latin1]{inputenc}

\def\N{\mathbb{N}}
\def\Z{\mathbb{Z}}
\def\Q{\mathbb{Q}}
\def\R{\mathbb{R}}
\def\C{\mathbb{C}}
\def\K{\mathbb{K}}

\def\diam{\mathrm{diam}}
\def\dom{\mathrm{dom}}
\def\card{\mathrm{card}}
\def\Re{\mathrm{Re}}
\def\Im{\mathrm{Im}}
\def\lin{\mathrm{lin}}
\def\dim{\mathrm{dim}}
\def\codim{\mathrm{codim}}
\def\co{\mathrm{co}}
\def\Int{\mathrm{Int}}

\usepackage{vmargin}
\setmargins{2.5cm}       % margen izquierdo
{1.5cm}                        % margen superior
{15.5cm}                      % anchura del texto
{23.42cm}                    % altura del texto
{10pt}                           % altura de los encabezados
{1cm}                           % espacio entre el texto y los encabezados
{0pt}                             % altura del pie de p�gina
{2cm}                           % espacio entre el texto y el pie de p�gina
\begin{document}

\begin{center}
\textbf{Problemas de An\'{a}lisis Funcional Avanzado}

\textbf{Tema 1: Principios fundamentales del An\'{a}lisis Funcional (repaso)}

\bigskip

David Cabezas Berrido

\end{center}

\bigskip

\textbf{Ejercicio 1:} Sean $X$ e $Y$ espacios de Banach y $T\colon X\to Y$ una aplicaci\'{o}n lineal y continua. Pruebe que las siguientes afirmaciones son equivalentes:
\begin{enumerate}
	\item $\ker(T)=\{0_X\}$ y $T(X)$ es cerrado en $Y$.
	\item Existe una constante $\alpha>0$ tal que $\alpha \left\|x\right\|\leq\left\|T(x)\right\|$ para todo $x\in X$.
\end{enumerate}

\bigskip

\textbf{Soluci\'on:}

	Supongamos (1) y consideremos $T$ como una aplicaci�n sobreyectiva sobre su imagen, $T:X\rightarrow T(X)$. Adem�s, la condici�n $\ker(T)=\{0_X\}$ nos dice que $T$ es inyectiva, luego tenemos una biyecci�n. Por otra parte, al ser $T(X)$ un subespacio cerrado de un espacio completo $Y$, deducimos que $T(X)$ es tambi�n un espacio de Banach.
	
	Aplicando el teorema de los isomorfismos de Banach concluimos que $T:X\rightarrow T(X)$ es un isomorfismo, por tanto $T^{-1}:T(X)\rightarrow X$ es continua. Esto equivale a que exista $M>0$ tal que
	\[\|T^{-1}(y)\|\leq M\|y\|\quad\forall y\in T(X).\]
	
	Como $T:X\rightarrow T(X)$ es una biyecci�n, decir $y\in T(X)$ es tan arbitrario como decir $T(x)$ con $x\in X$, luego tenemos
	\[\|x\|=\|T^{-1}(T(x))\|\leq M\|T(x)\|\quad\forall x\in X.\]
	Tomando $\alpha=M^{-1}>0$ obtenemos la condici�n deseada, (2).
	
	\smallskip
	
	Para la implicaci�n inversa, la desigualdad en (2) nos dice que si $x\in\ker(T)$, entonces $\alpha\|x\|\leq\|T(x)\|=0$, lo que fuerza (por ser $\alpha>0$) que $x=0$. Concluimos que $\ker(T)=\{0\}$ y que $T:X\rightarrow T(X)$ es una biyecci�n.
	
	Ahora tomamos una sucesi�n convergente cualquiera $\{y_n\}\to y\in Y$ de elementos de $T(X)$, nuestro objetivo es probar que $y\in T(X)$. Sea $\{x_n\}$ la sucesi�n de elementos de $X$ definida por $x_n=T^{-1}(y_n)$, la condici�n (2) nos dice que
	\[\|x_n-x_m\|\leq \alpha^{-1}\|T(x_n-x_m)\|=\alpha^{-1}\|y_n-y_m\| \quad\forall n,m\in\mathbb{N}.\]
	La sucesi�n $\{y_n\}$ es convergente y por tanto de Cauchy, luego la desigualdad anterior nos asegura que $\{x_n\}$ tambi�n es de Cauchy. Como $X$ es completo, existe $x\in X$ tal que $\{x_n\}\to x$. Finalmente, la continuidad de $T$ nos permite deducir que
	\[T(x)=T\left(\lim\limits_{n\to\infty}x_n\right)=\lim\limits_{n\to\infty}y_n=y.\]
	Concluimos que $y\in T(X)$, y de la arbitrariedad de la sucesi�n llegamos a que $T(X)$ es cerrado en $Y$.

\bigskip

\textbf{Ejercicio 2:}
En el contexto de espacios de Banach:
\begin{enumerate}
\item Pruebe el teorema de la aplicaci\'{o}n abierta usando el teorema de los isomorfismos de Banach.
\item Demuestre el teorema de la gr\'{a}fica cerrada usando el teorema de la aplicaci\'{o}n abierta.
\item Pruebe el teorema de los isomorfismos de Banach usando el teorema de la gr\'{a}fica cerrada.
\end{enumerate}

\bigskip

\textbf{Soluci\'on:}

A lo largo del ejercicio, $X$ e $Y$ ser�n dos espacios de Banach y $T:X\rightarrow Y$ una aplicaci�n lineal continua.

\begin{enumerate}
	\item Supongamos que $T$ es sobreyectiva, debemos probar que es abierta. Como $T$ es continua, $\ker(T)$ es cerrado en $Y$, luego el cociente $X/\ker(T)$ es un espacio de Banach. La descomposici�n can�nica nos asegura que la aplicaci�n $\tilde{T}:X/\ker(T)\rightarrow T(X)=Y$ dada por $\tilde{T}(x+\ker(T))=T(x)$ es una biyecci�n lineal continua. Por el teorema de los isomorfismos de Banach, $\tilde{T}$ es un isomorfismo, en particular abierta.
	\item 
\end{enumerate}

\bigskip

\textbf{Ejercicio 3:} Sea $\left\{y(n)\right\}_{n\in\N}$ una sucesi\'{o}n de n\'umeros reales o complejos tal que la serie $\sum_{n\in\N}x(n)y(n)$ es convergente para toda sucesi\'{o}n $\left\{x(n)\right\}_{n\in\N}\in c_0$. Compruebe que la serie $\sum_{n\in\N}\left|y(n)\right|$ converge.

\bigskip

\textbf{Soluci\'on:}

\bigskip

\textbf{Ejercicio 4:} 
Sean $X$ e $Y$ espacios normados y $x_0\in X\backslash \{0_X\}$. Dado $y_0\in Y$, pruebe que existe $T\in L(X,Y)$ tal que $T(x_0)=y_0$ y $\left\|T\right\| \left\| x_0\right\| =\left\| y_0\right\| $.

\bigskip

\textbf{Soluci\'on:}

\bigskip

\textbf{Ejercicio 5:} 
Sea $X$ un espacio normado separable. Pruebe que existe un subconjunto numerable de $X^{*}$ que separa los puntos de $X$.

\bigskip

\textbf{Soluci\'on:}

\bigskip

\textbf{Ejercicio 6:} 
Sea $X$ un espacio de Banach, $Y$ un espacio normado y $T\colon X\to Y$ una aplicaci\'{o}n lineal. Se define una nueva norma en $X$ mediante la expresi\'{o}n: 
\[
\Vert x\Vert _1=\Vert x\Vert +\Vert T(x)\Vert ,\quad \forall x\in X.
\]
Pruebe que las siguientes afirmaciones son equivalentes:
\begin{enumerate}
\item  $T$ es continua.
\item  $\Vert \cdot \Vert _1$ es equivalente a $\Vert \cdot \Vert $.
\item  $\Vert \cdot \Vert _1$ es una norma completa.
\end{enumerate}

\bigskip

\textbf{Soluci\'on:}

\bigskip

\end{document}

