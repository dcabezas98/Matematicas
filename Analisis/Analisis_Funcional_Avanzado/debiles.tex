\documentclass{amsart}
\usepackage{amssymb}
\usepackage{amsmath}
\usepackage[latin1]{inputenc}

\usepackage{parskip}
\usepackage{enumerate}

\usepackage{comment}

\def\diam{\mathrm{diam}}
\def\dom{\mathrm{dom}}
\def\card{\mathrm{card}}
\def\Re{\mathrm{Re}}
\def\Im{\mathrm{Im}}
\def\lin{\mathrm{lin}}
\def\dim{\mathrm{dim}}
\def\codim{\mathrm{codim}}
\def\co{\mathrm{co}}
\def\Int{\mathrm{Int}}

\setlength{\parindent}{0mm}

\usepackage{vmargin}
\setmargins{2.5cm}       % margen izquierdo
{1.5cm}                        % margen superior
{15.5cm}                      % anchura del texto
{23.42cm}                    % altura del texto
{10pt}                           % altura de los encabezados
{1cm}                           % espacio entre el texto y los encabezados
{0pt}                             % altura del pie de p�gina
{2cm}                           % espacio entre el texto y el pie de p�gina
\begin{document}

\begin{center}
\textbf{Problemas de An\'{a}lisis Funcional Avanzado}

\textbf{Tema 3: Topolog�as d�biles}

\smallskip

David Cabezas Berrido

\end{center}

\smallskip

\textbf{Ejercicio 1:} Sean $X,Y$ espacios normados. Caracterizar las aplicaciones lineales de $X$ en $Y$ que son $(n-w)$-continuas.

\smallskip

\textbf{Soluci\'on:}

Sea $T:X\rightarrow Y$ una aplicaci�n lineal. Sabemos por las propiedades de la topolog�a inicial y por la definici�n de la topolog�a d�bil que $T$ es $(n-w)$-continua si y solo si $y^*\circ T: X\rightarrow \mathbb{K}$ es continua (considerando la topolog�a $\tau_X$) para cada $y^*\in Y^*$. En tal caso, se tiene $y^*\circ T\in X^*$ para todo $y^*\in Y^*$ (la composici�n de aplicaciones lineales es claramente lineal). Por tanto, las aplicaciones lineales de $X$ en $Y$ que son $(n-w)$-continuas son aquellas $T:X\rightarrow Y$ que cumplen $y^*\circ T\in X^*$ para todo $y^*\in Y^*$.

\smallskip

\textbf{Ejercicio 2:} Sean $X$ un espacio normado, $Y$ un espacio de Banach. Caracterizar las aplicaciones lineales de $X$ en $Y^*$ que son $(n-w^*)$-continuas.

\smallskip

\textbf{Soluci\'on:} Sea $T:X\rightarrow Y^*$ una aplicaci�n lineal. Sabemos por las propiedades de la topolog�a inicial y por la definici�n de la topolog�a d�bil* que $T$ es $(n-w^*)$-continua si y solo si $\delta_y\circ T: X\rightarrow \mathbb{K}$ es continua (considerando la topolog�a $\tau_X$) para cada $\delta_y\in J_Y(Y)$. Fijado $y\in Y$ y dada una red en $X$ convergente a un punto de $X$ ($\{x_\lambda\}\to x\in X$), que $\delta_y\circ T$ sea continua en $x$ implica que la red $\{T(x_\lambda)(y)\}$ converge a $T(x)(y)$ en $\mathbb{K}$. Pero si esto pasa para toda red $\{x_\lambda\}\to x$, justamente significa que $\delta_y\circ T$ es continua en $x$. Por tanto, una aplicaci�n lineal $T:X\rightarrow Y^*$ es continua si y solo si la red de im�genes de toda red convergente en $X$ es una red de $Y^*$ puntualmente convergente a la imagen del l�mite. Por la linealidad, basta con que para toda red en $X$ convergente a $0$ (en la topolog�a de la norma), la red de im�genes converja puntualmente al funcional nulo $0\in Y^*$.

Otra forma de verlo: Como la composici�n de aplicaciones lineales es lineal, podemos asegurar que $T:X\rightarrow Y^*$ es continua si y solo si $\delta_y\circ T\in X^*$ para cada $\delta_y\in J_Y(Y)$.

\smallskip

\textbf{Ejercicio 3:} Sean $X,Y$ espacios normados. Caracterizar las aplicaciones lineales de $X$ en $Y$ que son $(w-n)$-continuas.

\smallskip

\textbf{Soluci\'on:} Sea $T:X\rightarrow Y$ una aplicaci�n lineal. Tomemos $\{x_\lambda\}$ una red en $X$ $w$-convergente a 0: $\{x^*(x_\lambda)\}\to 0 \ \ \forall x^*\in X^*$. Para que $T$ sea continua, es necesario que $\{T(x_\lambda)\}$ converja a 0 en la topolog�a de la norma de $Y$. Rec�procamente, $T$ es continua si esto le ocurre a toda red d�bilmente convergente a 0. Por tanto, $T$ es continua si y solo si para toda red $\{x_\lambda\}$ en $X$ satisfaciendo $\{x^*(x_\lambda)\}\to 0 \ \ \forall x^*\in X^*$, se tiene $\{\|T(x_\lambda)\|_Y\}\rightarrow 0$.

Otra forma de verlo: Como estamos tratando con topolog�as vectoriales, basta con que la preimagen de todo entorno de 0 en $Y$ sea $w$-entorno de 0 en $X$. Adem�s, como las homotecias son homeomorfismos, basta comprobarlo para la bola cerrada unidad de $Y$. Ser un entorno d�bil de 0 equivale a contener a un b�sico: que existan $\varepsilon>0$ y $f_1,\ldots,f_n\in X^*$ tales que
\[U(0,\varepsilon,f_1,\ldots,f_n)=\{x\in X:|f_k(x)|<\varepsilon \ \ \forall k=1,\ldots,n\}\subset T^{-1}(B_Y).\]
Por tanto, $T$ es continua si y solo si existen $\varepsilon>0$ y $f_1,\ldots,f_n\in X^*$ tales que
\[T\left(\bigcap_{k=1}^n f_k^{-1}\big(D(0,\varepsilon)\big)\right)\subset B_Y.\]

\smallskip

\textbf{Ejercicio 4:} Sean $X$ un espacio normado, $Y$ un espacio de Banach. Caracterizar las aplicaciones lineales de $X$ en $Y^*$ que son $(w-w^*)$-continuas.

\smallskip

\textbf{Soluci\'on:} Sea $T:X\rightarrow Y^*$ una aplicaci�n lineal. Sabemos por las propiedades de la topolog�a inicial y por la definici�n de la topolog�a d�bil* que $T$ es $(w-w^*)$-continua si y solo si $\delta_y\circ T: X\rightarrow \mathbb{K}$ es continua (considerando la topolog�a d�bil en $X$) para cada $\delta_y\in J_Y(Y)$. Como la composici�n de aplicaciones lineales es lineal, esto es lo mismo que decir que $\delta_y\circ T\in (X,w)^*=X^*$ para todo $\delta_y\in J_Y(Y)$. Por tanto, $T$ es $(w-w^*)$-continua si y solo si es $(n-w)$-continua.

\smallskip

\textbf{Ejercicio 5:} Sean $X,Y$ espacios normados. Caracterizar las aplicaciones lineales de $X^*$ en $Y$ que son $(w^*-n)$-continuas.

\smallskip

\textbf{Soluci\'on:} Sea $T:X^*\rightarrow Y$ una aplicaci�n lineal. $T$ es continua si y solo si para cada red $\{x_\lambda\}$ $w^*$-convergente en $X^*$ (puntualmente convergente como red de funciones de $X$ en $\mathbb{K}$) al funcional nulo, la red de escalares $\{\|T(x_\lambda)\|\}$ converge a 0.

\smallskip

\textbf{Ejercicio 6:} Sean $X,Y$ espacios normados. Caracterizar las aplicaciones lineales de $X^*$ en $Y$ que son $(w^*-w)$-continuas.

\smallskip

\textbf{Soluci\'on:} Sea $T:X^*\rightarrow Y$ una aplicaci�n lineal. Sabemos por las propiedades de la topolog�a inicial y por la definici�n de la topolog�a d�bil que $T$ es $(w^*-w)$-continua si y solo si $y^*\circ T: X^*\rightarrow \mathbb{K}$ es continua (considerando la topolog�a d�bil* en $X^*$) para cada $y^*\in Y^*$. Como la composici�n de aplicaciones lineales es lineal, esto equivale a que $y^*\circ T\in (X^*,w^*)^*=J_X(X)$ para cada $y^*\in Y^*$. Por tanto, $T$ es continua si y solo si $y^*\circ T\in (X^*,w^*)^*=J_X(X)$ para todo $y^*\in Y^*$.

\end{document}