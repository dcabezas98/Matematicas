\documentclass[12pt,spanish]{article}

% aprovechamiento de la p\'agina -- fill an A4 (210mm x 297mm) page
% Note: 1 inch = 25.4 mm = 72.27 pt
% 1 pt = 3.5 mm (approx)

% vertical page layout -- one inch margin top and bottom
\topmargin      -10 mm   % top margin less 1 inch
\headheight       0 mm   % height of box containing the head
\headsep          0 mm   % space between the head and the body of the page
\textheight     255 mm   % the height of text on the page
\footskip         7 mm   % distance from bottom of body to bottom of foot

% horizontal page layout -- one inch margin each side
\oddsidemargin    0 mm     % inner margin less one inch on odd pages
\evensidemargin   0 mm     % inner margin less one inch on even pages
\textwidth      159 mm     % normal width of text on page

\usepackage{parskip}

\usepackage[document]{ragged2e}
\usepackage{babel}
\usepackage[utf8]{inputenc}
\usepackage{amsmath,amsthm,mathtools}
\usepackage{amsfonts,amssymb,latexsym}
\usepackage{enumerate}
\usepackage[dvips,usenames]{color}
\definecolor{RojoAnayelRey}{rgb}{1,.25,.25}
\usepackage{tikz}
\usepackage[bookmarks=true,
            bookmarksnumbered=false, % true means bookmarks in 
                                     % left window are numbered                         
            bookmarksopen=false,     % true means only level 1
                                     % are displayed.
            colorlinks=true,
            linkcolor=webred]{hyperref}
\definecolor{webgreen}{rgb}{0, 0.5, 0} % less intense green
\definecolor{webblue}{rgb}{0, 0, 0.5}  % less intense blue
\definecolor{webred}{rgb}{0.5, 0, 0}   % less intense red
\definecolor{dkgreen}{rgb}{0,0.6,0}
\definecolor{gray}{rgb}{0.5,0.5,0.5}
\definecolor{mauve}{rgb}{0.58,0,0.82}
\definecolor{MistyRose}{RGB}{255,228,225}
\definecolor{LightCyan}{RGB}{224,255,255}

%\usepackage{beton}
\usepackage[T1]{fontenc}

% Theorem environments

%% \theoremstyle{plain} %% This is the default
\newtheorem{theorem}{Teorema}[section]
\newtheorem{corollary}[theorem]{Corolario}
\newtheorem{lemma}[theorem]{Lema}
\newtheorem{proposition}[theorem]{Proposici\'on}
%\newtheorem{ax}{Axioma}

\theoremstyle{definition}
\newtheorem{definition}{Definici\'on}[section]
\newtheorem{algorithm}{\textrm{\bf Algoritmo}}[section]

\theoremstyle{remark}
\newtheorem{remark}{Observaci\'on}[section]
\newtheorem{example}{Ejemplo}[section]
\newtheorem{exercise}{Ejercicio}[section]
%\newenvironment{solution}{\begin{proof}[Solution]}{\end{proof}}
\newenvironment{solution}{\begin{proof}[Solución]}{\end{proof}}
\newtheorem*{notation}{Notaci\'on}

%\numberwithin{equation}{section}

%\newcommand{\regla}[2]{
%\begin{array}{c}
%#1\\
%\hline
%#2\\
%\end{array}
%}

\title{Principales resultados de Análisis Vectorial}

\author{David Cabezas Berrido}

\date{}

\begin{document}

\maketitle

Vamos a presentar los tres principales resultados de la asignatura Análisis Vectorial: el Teorema de Green, el Teorema de la Stokes y el Teorema de la Divergencia. Todos ellos establecen propiedades relativas a las integrales en curvas y superficies de funciones reales de varias variables, normalmente en dimensión 2 y 3.

A continuación, introducimos los conceptos necesarios para la comprensión de cada uno de estos resultados. Seguidamente, iremos presentando los teoremas principales, en cada caso incluiremos una breve explicación intuitiva y un ejemplo donde comprobaremos que se verifica el resultado.

\section{Conceptos previos}

\subsection{Operadores diferenciales}

Los siguientes operadores diferenciales tienen una gran importancia en el análisis real, y algunos de ellos son necesarios para comprender los teoremas antes mencionados.

\subsubsection*{Gradiente}

Sea $f:\Omega\rightarrow\mathbb{R}$ un campo escalar diferenciable, donde $\Omega$ es un abierto de $\mathbb{R}^N$. El \textbf{gradiente} de $f$ es la función $\nabla f: \Omega\rightarrow\mathbb{R}^N$) dada por el vector de derivadas parciales, esto es,
\[\nabla f(x)=\left(\frac{\partial}{\partial x_1}f(x),\ldots,\frac{\partial}{\partial x_1}f(x)\right)\]
para cada $x\in \Omega$.

\subsubsection*{Divergencia}

Si $F:\Omega\rightarrow\mathbb{R}^N$ es un campo vectorial diferencible dado por $F=(F_1,\ldots, F_N)$, la \textbf{divergencia} de $F$ viene dada por
\[\operatorname{div}\big(F(x)\big)=\sum_{i=1}^N\frac{\partial}{\partial x_i}F_i(x).\]
Se tiene $\operatorname{div} F:\Omega\rightarrow\mathbb{R}$.

\subsubsection*{Laplaciano}

Si ahora suponemos que $f$ es de clase 2 ($f\in C^2(\Omega)$), definimos el \textbf{laplaciano} de $f$ como la traza de su matriz Hessiana, es decir, la función $\Delta f:\Omega\rightarrow\mathbb{R}$ dada por
\[\Delta f(x)=\sum_{i=1}^{N}\frac{\partial^2}{\partial x_i^2}f(x).\]
Además, se tiene $\Delta f=\operatorname{div}(\nabla f)$.

\subsubsection*{Rotacional}

Cuando la dimensión es $N=3$, definimos la \textbf{rotacional} del campo vectorial diferenciable $F=(F_1,F_2,F_3)$ como el campo (vectorial) dado por
\[\operatorname{rot F}=\left(\frac{\partial F_3}{\partial x_2}-\frac{\partial F_2}{\partial x_3},\frac{\partial F_1}{\partial x_3}-\frac{\partial F_3}{\partial x_1},\frac{\partial F_2}{\partial x_1}-\frac{\partial F_1}{\partial x_2}\right).\]

Para el caso bidimensional, donde $F=(F_1,F_2)$, tenemos
\[\operatorname{rot F}=\frac{\partial F_2}{\partial x_1}-\frac{\partial F_1}{\partial x_2}.\]
Notemos que en este caso, la rotacional es un campo escalar.

\subsection{Integrales de campos escalares y vectoriales sobre curvas y superficies}

Con la notación anterior, introducimos la forma de integrar campos escalares y vectoriales sobre curvas y superficies.

\subsubsection*{Integral de línea de un campo escalar}

Si $f:\Omega\rightarrow\mathbb{R}$ es un campo escalar continuo con $\Omega\subset\mathbb{R}^N$ y $\gamma:[a,b]\rightarrow\Omega$ es un camino regular a trozos (una función de clase $C^1$ a trozos del intervalo $[a,b]$ en $\Omega$), definimos la \emph{integral de línea de $f$ a lo largo de $\gamma$} como
	\[\int_\gamma f dl=\int_a^b f\big(\gamma(t)\big)\|\gamma'(t)\|dt.\]
	
\subsubsection*{Integral de línea de un campo vectorial}

Si $F:\Omega\rightarrow\mathbb{R}^N$ es un campo escalar continuo con $\Omega\subset\mathbb{R}^N$ y $\gamma:[a,b]\rightarrow\Omega$ es un camino regular a trozos, definimos la \emph{integral de línea de $F$ a lo largo de $\gamma$} como
\[\int_\gamma F dl=\int_a^b \left\langle F\big(\gamma(t)\big)\middle|\gamma'(t)\right\rangle dt.\]

TODO: superficies

\section{Teorema de Green}

TODO: pag 11

\section{Teorema de la Divergencia en $\mathbb{R}^2$}

TODO: pag 13

\section{Teorema de Stokes}

TODO: pag 19

\section{Teorema de la divergencia en $\mathbb{R}^N$}

\subsection{Preámbulo}

TODO: dominio regular (se puede hablar de normal exterior a la frontera).

En un regular se puede despejar en la frontera una componente en función de las demás.

Particiones continuas de la unidad, caso compacto (cierre de acotado)

\subsection{Resultado pricipal}

TODO: pag 26

\end{document}

