\documentclass{amsart}
\usepackage{amssymb}
\usepackage{amsmath}
\usepackage[latin1]{inputenc}

\usepackage{parskip}

\def\N{\mathbb{N}}
\def\Z{\mathbb{Z}}
\def\Q{\mathbb{Q}}
\def\R{\mathbb{R}}
\def\C{\mathbb{C}}
\def\K{\mathbb{K}}

\def\diam{\mathrm{diam}}
\def\dom{\mathrm{dom}}
\def\card{\mathrm{card}}
\def\Re{\mathrm{Re}}
\def\Im{\mathrm{Im}}
\def\lin{\mathrm{lin}}
\def\dim{\mathrm{dim}}
\def\codim{\mathrm{codim}}
\def\co{\mathrm{co}}
\def\Int{\mathrm{Int}}

\usepackage{enumerate}

\usepackage{vmargin}
\setmargins{2.5cm}       % margen izquierdo
{1.5cm}                        % margen superior
{15.5cm}                      % anchura del texto
{23.42cm}                    % altura del texto
{10pt}                           % altura de los encabezados
{1cm}                           % espacio entre el texto y los encabezados
{0pt}                             % altura del pie de p�gina
{2cm}                           % espacio entre el texto y el pie de p�gina
\begin{document}

\begin{center}
\textbf{Problemas de An�lisis Matem�tico Avanzado}

\textbf{Tema 1: Funciones Arm�nicas}

\bigskip

David Cabezas Berrido

\end{center}

\bigskip

\textbf{Ejercicio 1:}
\begin{enumerate}[a)]
	\item Ecuaciones de Cauchy-Riemann en polares para $u^*(r,\theta)=u(re^{i\theta})$ y $v^*(r,\theta)=v(re^{i\theta})$.
	\item Ecuacion de Laplace en coordenadas polares para $u^*(r,\theta)=u(re^{i\theta})$.
\end{enumerate}

\bigskip

\textbf{Soluci\'on:}

Tenemos $u^*(r,\theta)=u(r\cos\theta, r\sin\theta)$. Por la regla de la cadena
\begin{equation*}
\frac{\partial u^*}{\partial r}(r,\theta)=\frac{\partial u}{\partial r}(r\cos\theta,r\sin\theta)=\frac{\partial u}{\partial x}\frac{\partial (r\cos\theta)}{\partial r}+\frac{\partial u}{\partial y}\frac{\partial (r\sin\theta)}{\partial r}
=\frac{\partial u}{\partial x}\cos\theta+\frac{\partial u}{\partial y}\sin\theta,
\end{equation*}
\begin{equation*}
\frac{\partial u^*}{\partial \theta}(r,\theta)=\frac{\partial u}{\partial \theta}(r\cos\theta,r\sin\theta)=\frac{\partial u}{\partial x}\frac{\partial (r\cos\theta)}{\partial \theta}+\frac{\partial u}{\partial y}\frac{\partial (r\sin\theta)}{\partial \theta}
=\frac{\partial u}{\partial x}(-r\sin\theta)+\frac{\partial u}{\partial y}r\cos\theta.
\end{equation*}
Luego,
\begin{equation} \label{eq:e1-u*}
\frac{\partial u^*}{\partial r}(r,\theta)=\frac{\partial u}{\partial x}\cos\theta+\frac{\partial u}{\partial y}\sin\theta, \quad
\frac{\partial u^*}{\partial \theta}(r,\theta)=\frac{\partial u}{\partial x}(-r\sin\theta)+\frac{\partial u}{\partial y}r\cos\theta.
\end{equation}

An�logamente,
\begin{equation} \label{eq:e1-v*}
\frac{\partial v^*}{\partial r}(r,\theta)=\frac{\partial v}{\partial x}\cos\theta+\frac{\partial v}{\partial y}\sin\theta, \quad
\frac{\partial v^*}{\partial \theta}(r,\theta)=\frac{\partial v}{\partial x}(-r\sin\theta)+\frac{\partial v}{\partial y}r\cos\theta.
\end{equation}

Rec�procamente, recordando las relaciones entre $(x,y)$ y $(r,\theta)$ tenemos
\begin{equation*} 
\frac{\partial u}{\partial x}=\frac{\partial u^*}{\partial r}\frac{\partial r}{\partial x}+\frac{\partial u^*}{\partial \theta}\frac{\partial \theta}{\partial x}=\frac{\partial u^*}{\partial r}\cos\theta-\frac{\partial u^*}{\partial \theta}\frac{\sin\theta}{r},
\end{equation*}
\begin{equation*} 
\frac{\partial u}{\partial y}=\frac{\partial u^*}{\partial r}\frac{\partial r}{\partial y}+\frac{\partial u^*}{\partial \theta}\frac{\partial \theta}{\partial y}=\frac{\partial u^*}{\partial r}\sin\theta+\frac{\partial u^*}{\partial \theta}\frac{\cos\theta}{r}.
\end{equation*}

Por tanto, 
\begin{equation}  \label{eq:e1-u}
\frac{\partial u}{\partial x}=\frac{\partial u^*}{\partial r}\cos\theta-\frac{\partial u^*}{\partial \theta}\frac{\sin\theta}{r},\quad
\frac{\partial u}{\partial y}=\frac{\partial u^*}{\partial r}\sin\theta+\frac{\partial u^*}{\partial \theta}\frac{\cos\theta}{r}.
\end{equation}

\begin{enumerate}[a)]
	\item Sustituyendo las ecuaciones de Cauchy-Riemann en \eqref{eq:e1-u*} y \eqref{eq:e1-v*} tenemos
	\[\frac{\partial v^*}{\partial \theta}=\frac{\partial v}{\partial x}(-r\sin\theta)+\frac{\partial v}{\partial y}r\cos\theta=r\left(\frac{\partial u}{\partial y}\sin\theta+\frac{\partial u}{\partial x}\cos\theta\right)=r\frac{\partial u^*}{\partial r},\]
	
	\[\frac{\partial u^*}{\partial \theta}=\frac{\partial u}{\partial x}(-r\sin\theta)+\frac{\partial u}{\partial y}r\cos\theta=-r\left(\frac{\partial v}{\partial y}\sin\theta+\frac{\partial v}{\partial x}\cos\theta\right)=-r\frac{\partial v^*}{\partial r}.\]
	
	Las propiedades del cambio de variable a coordenadas polares nos dicen que estas ecuaciones son equivalentes a las ecuaciones cl�sicas de Cauchy-Riemann. Tambi�n se pueden sustituir las ecuaciones del enunciado en \eqref{eq:e1-u} y su an�loga para $v$.
	
	\item Sacamos a partir de \eqref{eq:e1-u} las derivadas parciales del laplaciano.
	\begin{align*}
	\frac{\partial^2 u}{\partial x^2}&=\frac{\partial}{\partial x}\left(\frac{\partial u^*}{\partial r}\cos\theta-\frac{\partial u^*}{\partial \theta}\frac{\sin\theta}{r}\right)\\
	&=\cos^2\theta \frac{\partial^2 u^*}{\partial r^2}-\frac{\partial^2 u^*}{\partial \theta\partial r}\cos\theta\frac{\sin\theta}{r}+\frac{\partial u^*}{\partial r}\left(\frac{1}{r}-\frac{\cos^2\theta}{r}\right) \\
	&-\frac{\sin\theta}{r}\frac{\partial^2 u^*}{\partial r\partial \theta}\cos\theta+\frac{\sin^2\theta}{r^2}\frac{\partial^2 u^*}{\partial \theta^2}+\frac{\partial u^*}{\partial \theta}\frac{2\cos\theta\sin\theta}{r^2} \\
	&=\frac{\partial^2 u^*}{\partial r^2}\cos^2\theta-\frac{\partial^2 u^*}{\partial \theta\partial r}\frac{2\cos\theta\sin\theta}{r}+\frac{\partial^2 u^*}{\partial \theta^2}\frac{\sin^2\theta}{r^2}+\frac{\partial u^*}{\partial r}\frac{\sin^2\theta}{r}+\frac{\partial u^*}{\partial \theta}\frac{2\cos\theta\sin\theta}{r^2}.
	\end{align*}
	
	\begin{align*}
	\frac{\partial^2 u}{\partial^2 y}&=\frac{\partial}{\partial y}\left(\frac{\partial u^*}{\partial r}\sin\theta+\frac{\partial u^*}{\partial \theta}\frac{\cos\theta}{r}\right) \\
	&= \sin^2\theta \frac{\partial^2 u^*}{\partial r^2}+\sin\theta \frac{\partial^2 u^*}{\partial\theta\partial r}\frac{\cos\theta}{r}+\frac{\partial u^*}{\partial r}\frac{\cos^2\theta}{r} \\
	&+\frac{\cos\theta}{r}\frac{\partial^2 u^*}{\partial r\partial \theta}\sin\theta+\frac{\cos^2\theta}{r^2}\frac{\partial^2 u^*}{\partial \theta^2}-\frac{\partial u^*}{\partial \theta}\frac{2\cos\theta\sin\theta}{r^2} \\
	&=\frac{\partial^2 u^*}{\partial r^2}\sin^2\theta +\frac{\partial^2 u^*}{\partial\theta\partial r}\frac{2\cos\theta\sin\theta}{r}+\frac{\partial^2 u^*}{\partial \theta^2}\frac{\cos^2\theta}{r^2}+\frac{\partial u^*}{\partial r}\frac{\cos^2\theta}{r}-\frac{\partial u^*}{\partial \theta}\frac{2\cos\theta\sin\theta}{r^2}.
	\end{align*}
	
	La ecuaci�n de Laplace queda as� tras agrupar, cancelar y simplificar algunos t�rminos
	\[0=\frac{\partial^2 u}{\partial x^2}+\frac{\partial^2 u}{\partial^2 y}=\frac{\partial^2 u^*}{\partial r^2}+\frac{1}{r^2}\frac{\partial^2 u^*}{\partial \theta^2}+\frac{1}{r}\frac{\partial u^*}{\partial r}.\]
\end{enumerate}

\bigskip

\textbf{Ejercicio 2:} Probar que $u(re^{i\theta})=\theta\log r$ es arm�nica en $\mathbb{C}\backslash\mathbb{R}^-$. Encontrar conjugada arm�nica $v$ de $u$. �Qu� funci�n es $f(z)=u(z)+iv(z)$.

\bigskip

\textbf{Soluci�n:}
Calculamos las derivadas:
\[\frac{\partial u}{\partial r}=\frac{\theta}{r},\quad \frac{\partial u}{\partial \theta}=\log r,\quad \frac{\partial^2 u}{\partial r^2}=-\frac{\theta}{r^2},\quad \frac{\partial^2 u}{\partial \theta^2}=0.\]
Sustituimos en la ecuaci�n de Laplace:
\[\frac{\partial^2 u}{\partial r^2}+\frac{1}{r^2}\frac{\partial^2 u}{\partial \theta^2}+\frac{1}{r}\frac{\partial u}{\partial r}=-\frac{\theta}{r^2}+\frac{1}{r^2}\cdot 0+\frac{1}{r}\frac{\theta}{r}=0.\]
Esto es v�lido en todo punto de $\mathbb{C}\backslash\mathbb{R}^-$, luego $u$ es arm�nica.

Sea $v$ la conjugada arm�nica de $u$, que existe porque el dominio es simplemente conexo. Imponemos condiciones para hallar $v$:
\[\frac{\partial v}{\partial \theta}=r\frac{\partial u}{\partial r}=r\frac{\theta}{r}=\theta,\]
\[\frac{\partial v}{\partial r}=-\frac{1}{r}\frac{\partial u}{\partial \theta}=-\frac{1}{r}\log r.\]
La primera ecuaci�n nos dice que $v$ es de la forma $v(r,\theta)=\dfrac{\theta^2}{2}+\phi(r)$, y la segunda que $v(r,\theta)=-\dfrac{\log^2 r}{2}+\psi(\theta)$. Por tanto podemos tomar $v(r,\theta)=-\dfrac{\log^2 r}{2}+\dfrac{\theta^2}{2}$.

Tomando $f(r e^{i\theta})=u(r e^{i\theta})+iv(r e^{i\theta})=\theta\log r+i\left(-\dfrac{\log^2 r}{2}+\dfrac{\theta^2}{2}\right)$ para todo $r e^{i\theta}\in \mathbb{C}\backslash\mathbb{R}^-$, tenemos $f\in\mathcal{H}(\mathbb{C}\backslash\mathbb{R}^-)$.

\bigskip

\textbf{Ejercicio 3:} 

\bigskip

\textbf{Soluci\'on:}

\bigskip

\end{document}

