\documentclass[12pt,english]{article}

% aprovechamiento de la p\'agina -- fill an A4 (210mm x 297mm) page
% Note: 1 inch = 25.4 mm = 72.27 pt
% 1 pt = 3.5 mm (approx)

% vertical page layout -- one inch margin top and bottom
\usepackage[left=3cm,right=3cm,top=2.5cm,bottom=2.5cm]{geometry}

\usepackage{parskip}

\usepackage{ bbm }

\usepackage[doument]{ragged2e}
\usepackage[spanish]{babel}
\usepackage[utf8]{inputenc}
\usepackage{amsmath,amsthm,mathtools}
\usepackage{amsfonts,amssymb,latexsym}
\usepackage{enumerate}
\usepackage[dvips,usenames]{color}
\usepackage{tikz}
\usepackage[bookmarks=true,
            bookmarksnumbered=false, % true means bookmarks in 
                                     % left window are numbered                         
           bookmarksopen=false,     % true means only level 1
                                     % are displayed.
            colorlinks=true,
            linkcolor=blue]{hyperref}

% Theorem environments

%% \theoremstyle{plain} %% This is the default
\newtheorem{theorem}{Teorema}
\newtheorem{corollary}[theorem]{Corollary}
\newtheorem{lemma}[theorem]{Lema}
\newtheorem{proposition}[theorem]{Proposition}
%\newtheorem{ax}{Axioma}

\newtheorem*{lemma*}{Lemma}

\theoremstyle{definition}
\newtheorem{definition}{Definition}[section]

\theoremstyle{remark}
\newtheorem{example}{Example}[section]

%\numberwithin{equation}{section}

\newcommand{\co}[1]{\operatorname{co}(#1)}
\newcommand{\cloco}[1]{\overline{\operatorname{co}}(#1)}

\title{Teorema de Caracterización de Pesos $A_1$}

\author{David Cabezas Berrido}

\date{}

\begin{document}

\maketitle

\section*{Introducción}

Vamos a demostrar el teorema de caracterización de los pesos $A_1$. Nuestra referencia principal será el libro ``Análisis de Fourier'' de Javier Duoandikoetxea. Fijemos primero algo de notación.

Trabajaremos en el espacio $\mathbb{R}^n$. En adelante $w$ denotará un peso, es decir, una función medible, no negativa y localmente integrable en $\mathbb{R}^n$. Para cada conjunto medible $E\subset \mathbb{R}^n$, notaremos $w(E)=\int_E w dx$, donde la integral es respecto a la medida de Lebesgue en $\mathbb{R}^n$. La medida de Lebesgue de un conjunto medible $E$ se denota por $|E|$.

Consideramos el funcional maximal de Hardy-Littlewood $M$ definido por
\begin{equation}\label{eq: H-L}
Mf(x)=\sup_{Q\ni x}\frac{1}{|Q|}\int_Q|f(y)|dy
\end{equation}
para cada $f$ localmente integrable en $\mathbb{R}^n$ \big($f\in L^1_{\rm loc}(\mathbb{R}^n)$\big). El supremo de la expresión de arriba es en todos los cubos diádicos $Q$ que contienen al punto $x\in\mathbb{R}^n$.

La condición para que un peso $w$ esté en la clase $A_1$ es
\begin{equation}\label{eq: A1}
\frac{w(Q)}{|Q|}\leq C w(x) 
\end{equation}
para casi todo $x\in Q$ y para todo cubo diádico $Q$. La constance $C$ no puede depender ni de $x$ ni de $Q$, se le llama \emph{constante $A_1$ de $w$}.

\section*{Demostración del teorema}

Primero enunciaremos dos resultados que necesitaremos para la prueba del teorema. El primero es la \emph{desigualdad de Kolmogorov}.

\begin{lemma} \label{lm: kolmogorov} Si $T$ es un operador $(1,1)$-débil y $\delta\in [0,1[$, se tiene
	\[\int_E |Tf|^\delta dx\leq C(\delta) |E|^{1-\delta}\|f\|_{1}^\delta\]
	para alguna constante $C(\delta)$ dependiente de $\delta$ válida para toda $f$ integrable.
\end{lemma}

Sabemos que el operador $M$ es (1,1)-débil, por lo que podremos aplicarle éste resultado. El siguiente es la \emph{desigualdad de Hölder inversa}.

\begin{lemma} \label{lm: inverse holder} Si $w\in A_p$ con $1<p<\infty$. Existe $\varepsilon>0$ dependiente sólo de $p$ y de la constante $A_p$ de $w$ tal que
	\[\left(\frac{1}{|Q|}\int_Q w^{1+\varepsilon}\right)^{\frac{1}{1+\varepsilon}}\leq \frac{C}{|Q|}\int_Q w,\]
	donde la constante $C$ es válida para todo cubo diádico $Q$.
\end{lemma}

Ya estamos en condiciones de demostrar el teorema de caracterización de pesos $A_1$.

\begin{theorem}\label{thm: A1 characterization} Sea $f\in L^1_{\rm loc}(\mathbb{R}^n)$ tal que $Mf(x)<\infty$ casi por doquier en $\mathbb{R}^n$. Si $\delta\in[0,1[$, $w(x)=\big(Mf(x)\big)^\delta$ es un peso $A_1$ con constante $A_1$ dependiente del $\delta$ pero no de $f$.
	
Recíprocamente, si $w\in A_1$ existen $f\in L^1_{\rm loc}(\mathbb{R}^n)$, $k\in L^\infty$ con $k^{-1}\in L^\infty$ y $\delta\in[0,1[$ tales que $w=k(Mf)^\delta$.
\end{theorem}

\begin{proof}
	Para la primera parte, debemos probar que para todo cubo diádico $Q$ y para casi todo $x\in Q$ se tiene la condición $A_1$:
	\[\frac{1}{|Q|}\int_Q(Mf)^\delta\leq C\big(Mf(x)\big)^\delta\]
	con $C$ independientemente de $Q$ y de $f$. Fijados $Q$ y $f$, sea $\overline{Q}$ el cubo con el mismo centro y el doble de lado. De esta forma, $\overline{Q}$ también es un cubo diádico con $|\overline{Q}|=2^n|Q|$. Podemos escribir $f=f_1+f_2$ con $f_1=f\cdot\chi_{\overline{Q}}$ y $f_2=f-f_1$. Tenemos
	\begin{align*}
	Mf(x)&=\sup_{Q\ni x}\frac{1}{|Q|}\int_Q|f_1(y)+f_2(y)|dy\leq\sup_{Q\ni x}\left(\frac{1}{|Q|}\int_Q|f_1(y)|dy+\frac{1}{|Q|}\int_Q|f_2(y)|dy\right) \\
	&\leq\sup_{Q\ni x}\frac{1}{|Q|}\int_Q|f_1(y)|dy+\sup_{Q\ni x}\frac{1}{|Q|}\int_Q|f_2(y)|dy=Mf_1(x)+Mf_2(x)
	\end{align*}
	para casi todo $x\in\mathbb{R}^n$. Por tanto, si $\delta\in [0,1[$, $Mf(x)^\delta\leq Mf_1(x)^\delta+Mf_2(x)^\delta$ pct (para casi todo) $x$.
	
	Trabajemos primero con $f_1\in L^1(\mathbb{R}^n)$. Usando el Lema \ref{lm: kolmogorov}, puesto que $M$ es (1,1)-débil y no negativo, obtenemos
	\begin{equation} \label{eq: cota int Mf1}
	\begin{split}
		\frac{1}{|Q|}\int_Q(Mf_1)^\delta &\leq\frac{1}{|Q|} C(\delta)|Q|^{1-\delta}\|f_1\|_1^\delta=C(\delta)\left(\frac{\int_{\overline{Q}}|f|}{|Q|}\right)^\delta =C(\delta)\left(\frac{\int_{\overline{Q}}|f|}{|\overline{Q}|/2^n}\right)^\delta \\ &\leq C(\delta)2^{\delta n} \left(\sup_{Q\ni x}\frac{1}{|Q|}\int_Q|f|\right)^\delta\leq C(\delta)2^{n} Mf(x)^\delta
	\end{split}
	\end{equation}
	pct $x\in\overline{Q}$, en particular, pct $x\in Q$. En el segundo paso hemos usado que $f_1=f\cdot\chi_{\overline{Q}}$.
	
	Por otra parte,
\end{proof}

\begin{thebibliography}{99}
\bibitem{duo} J. Duoandikoetxea: \emph{Análisis de Fourier}. Universidad Autónoma de Madrid, 1995.
\end{thebibliography}

\end{document}
