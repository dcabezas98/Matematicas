\documentclass{article}
\usepackage[left=3cm,right=3cm,top=2cm,bottom=2cm]{geometry} % page settings

\usepackage{amsmath,amssymb,amsthm,amsfonts}
\usepackage{mathtools}

\usepackage[spanish]{babel}

\usepackage{multirow}

\usepackage[utf8]{inputenc}
\setlength{\parindent}{0mm}

\usepackage[parfill]{parskip}

% \usepackage{beton}
% \usepackage{euler} 
% \usepackage[T1]{fontenc}

\newtheorem*{theorem}{}

% Entorno para estilo de ejercicios
\newenvironment{ejercicio}[1]{\textbf{#1} \vspace*{5mm}}{\vspace*{5mm}}

\begin{document}

\title{\Large Probar que $g(x)=\dfrac{-x}{x^3+\cos^2x}$ es integrable en $(0,+\infty)$}

\author{David Cabezas}
\date{}
\maketitle

\begin{displaymath}
  \int_0^\infty g(x)dx = \int_0^1 g(x)dx + \int_1^\infty g(x)dx = \int_0^1 \frac{-x}{x^3+\cos^2x}dx + \int_1^\infty \frac{-x}{x^3+\cos^2x}dx
\end{displaymath}

Trabajaremos con ambos sumandos por separado, debemos probar que ambos
son finitos.

La función $\dfrac{-x}{x^3+\cos^2x}$ es continua en $[0,1]$, por tanto
es integrable, luego el primer sumando está acotado.

Respecto al segundo, notemos que $\forall x \in (0,+\infty)$.

\begin{displaymath}
  0 > \frac{-x}{x^3+\cos^2x} \geq \frac{-1}{x^2}
\end{displaymath}

Luego se tiene

\begin{displaymath}
  0 > \int_1^\infty \frac{-x}{x^3+\cos^2x}dx \geq \int_1^\infty \frac{-1}{x^2}dx = \frac{1}{x}\bigg]_1^\infty = -1
\end{displaymath}

Por tanto, ambos sumandos son finitos y $g$ es integrable.

\end{document}