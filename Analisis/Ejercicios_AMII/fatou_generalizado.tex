\documentclass{article}
\usepackage[left=3cm,right=3cm,top=2cm,bottom=2cm]{geometry} % page settings

\usepackage{amsmath,amssymb,amsthm,amsfonts}
\usepackage{mathtools}

\usepackage[spanish]{babel}

\usepackage{multirow}

\usepackage[utf8]{inputenc}
\setlength{\parindent}{0mm}

\usepackage[parfill]{parskip}

\usepackage{beton}
\usepackage{euler} 
\usepackage[T1]{fontenc}

\newtheorem*{theorem}{}

% Entorno para estilo de ejercicios
\newenvironment{ejercicio}[1]{\textbf{#1} \vspace*{5mm}}{\vspace*{5mm}}

\begin{document}

\title{\Large Generalización del Lema de Fatou}
\author{David Cabezas}
\date{}
\maketitle

\begin{theorem}[\textbf{Lema de Fatou para funciones positivas}]
  Sea $(\Omega,\mathcal{A},\mu)$ un espacio de medida y
  $E \in \mathcal{A}$.

  Si $f_n:\Omega \longrightarrow [0,\infty]$ es una sucesión de
  funciones medibles entonces se cumple que
  \[\int_E\varliminf\limits_{n\to\infty} f_n d\mu \leq
    \varliminf\limits_{n\to\infty} \int_Ef_nd\mu\]
\end{theorem}

\begin{theorem}[\textbf{Lema de Fatou generalizado}]
  Sea $(\Omega,\mathcal{A},\mu)$ un espacio de medida y
  $E \in \mathcal{A}$.

  Si $f_n:\Omega \longrightarrow \mathbb{R}$ es una sucesión de
  funciones medibles y $\exists h:\Omega \longrightarrow \mathbb{R}$
  integrable cumpliendo $h(\omega) \leq f_n(\omega)$ a.e.
  $\omega \in E \hspace{2mm} \forall n \in \mathbb{N}$ entonces se
  cumple que
  \[\int_E\varliminf\limits_{n\to\infty} f_n d\mu \leq
    \varliminf\limits_{n\to\infty} \int_Ef_nd\mu\]
\end{theorem}

\textbf{\underline{Demostración:}}

Para cada $n \in \mathbb{N}$, definimos
$A_n:=\{\omega \in E : h(\omega) > f_n(\omega)\}$, que es medible. \\
Sabemos que $\mu(A_n)=0 \ \forall n \in \mathbb{N}$ ya que
$h(\omega) \leq f_n(\omega)$ a.e. en $E$.

Tomamos $A=E\Big\backslash\bigcup\limits_{n=1}^\infty A_n$, que tiene
medida nula. Dado $\omega \in A$ se tiene
$h(\omega) \leq f_n(\omega) \hspace{2mm} \forall n \in \mathbb{N}$.

Ahora consideramos la sucesión de funciones
$g_n:\Omega \longrightarrow \mathbb{R}$, donde
$g_n=(f_n-h)\mathcal{X}_A \hspace{2mm} \forall n \in \mathbb{N}$, que
son medibles.

Dado $\omega \in E$, $g_n(\omega)=f_n(\omega)-h(\omega)$ si
$\omega \in A$ y $g_n(\omega)=0$ si $\omega \notin A$. En cualquier
caso se tiene $g_n(\omega) \leq 0$. Por lo que podemos aplicar el
resultado anterior a la sucesión $\{g_n\}$, obteniendo
\begin{align}
  &\int_E\varliminf\limits_{n\to\infty} g_n d\mu \leq \varliminf\limits_{n\to\infty} \int_Eg_nd\mu \\
  &\int_E\varliminf\limits_{n\to\infty} (f_n-h)\mathcal{X}_A d\mu \leq \varliminf\limits_{n\to\infty} \int_E(f_n-h)\mathcal{X}_A d\mu \\
  &\int_E\varliminf\limits_{n\to\infty} f_n\mathcal{X}_A - h\mathcal{X}_A d\mu \leq \varliminf\limits_{n\to\infty} \int_E(f_n-h)\mathcal{X}_A d\mu = \varliminf\limits_{n\to\infty} \int_Ef_n-h d\mu \\
  &\int_E \varliminf\limits_{n\to\infty} f_n\mathcal{X}_A d\mu - \int_Eh\mathcal{X}_A d\mu = \int_E\varliminf\limits_{n\to\infty} f_n\mathcal{X}_A - h\mathcal{X}_A d\mu \leq \varliminf\limits_{n\to\infty} \int_Ef_n d\mu - \int_Eh d\mu \\
  &\int_A \varliminf\limits_{n\to\infty} f_n d\mu = \int_E \varliminf\limits_{n\to\infty} f_n\mathcal{X}_A d\mu \leq \varliminf\limits_{n\to\infty} \int_Ef_n d\mu
\end{align}

Donde en (3) hemos usado que
$\int_E(f_n-h)\mathcal{X}_A d\mu = \int_Ef_n-h d\mu$ ya que
$(f_n-h)\mathcal{X}_A = f_n-h$ a.e. en $E$. \\
En (4) hemos usado que $\int_Eh d\mu < \infty$ ($h$ integrable) para
separar las integrales.

Finalmente tenenemos

\[\int_E \varliminf\limits_{n\to\infty} f_n = \int_A \varliminf\limits_{n\to\infty} f_n + \int_{E\backslash A}\varliminf\limits_{n\to\infty} f_n = \int_A \varliminf\limits_{n\to\infty} f_n + 0\]

Ya que $\mu(E\backslash A)=0$.

\hfill\qedsymbol

\end{document}