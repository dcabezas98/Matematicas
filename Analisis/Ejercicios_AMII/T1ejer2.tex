\documentclass{article}
\usepackage[left=3cm,right=3cm,top=2cm,bottom=2cm]{geometry} % page settings
\usepackage{amsmath} % provides many mathematical environments & tools
\usepackage{amssymb}
\usepackage{amsfonts}
\usepackage[spanish]{babel}



\usepackage{multirow}

\usepackage{algorithm}
\usepackage{algpseudocode}
\usepackage{pifont}

\usepackage[utf8]{inputenc}
\setlength{\parindent}{0mm}

\usepackage[parfill]{parskip}

% Para el código
\usepackage{listings}
\usepackage{xcolor}
\definecolor{gray}{rgb}{0.5,0.5,0.5}
\newcommand{\n}[1]{{\color{gray}#1}}
\lstset{numbers=left,numberstyle=\small\color{gray}}

% Entorno para estilo de ejercicios
\newenvironment{ejercicio}[1]{\textbf{#1} \vspace*{5mm}}{\vspace*{5mm}}
\setlength{\parindent}{10pt} 

\begin{document}

\title{Tema 1: Problema 2}
\author{David Cabezas}
\date{}
\maketitle
\begin{flushleft}
  Probar que el espacio normado de funciones continuas y
  acotadas $(\mathcal{C}_b(A,\mathbb{R}), \|\cdot\|_\infty)$ es
  de Banach, es decir, completo.

  En adelante $\|\cdot\|$ denota $\|\cdot\|_\infty$.

  Tomamos una sucesión de Cauchy $\{f_n\}$ de elementos del espacio.
  Tomando $\epsilon=1$, existe un $m \in \mathbb{N}$ tal que
  $$\forall p,q \in \mathbb{N} \text{ cumpliendo }  p,q \geq m \text{ se tiene } \|f_p - f_q\| < 1$$
  en particular
  $$\forall n \in \mathbb{N} \text{ cumpliendo } n \geq m \text{ se
    tiene } \|f_n - f_m\| < 1$$ Por tanto, $\|f_n\| - \|f_m\| \leq
  \|f_n - f_m\| < 1 \Rightarrow \|f_n\| \leq 1 + \|f_m\|$, $m$ es
  fijo, luego la sucesión está acotada:
  $$\|f_n\| \leq M \hspace{2mm} \forall n \in \mathbb{N}; \text{
    siendo } M = max\{1+\|f_m\|, \{\|f_k\| : k \in \mathbb{N}, k
  \leq m\}\}$$

  Tomamos $F$ como la bola cerrada de radio $M$ centrada en el origen,
  $F$ es cerrado en $\mathcal{C}_b(A,\mathbb{R})$ y acotado, por tanto
  compacto. Además, $\{f_n\} \subset F$. Por tanto existe una parcial
  $\{f_{\sigma(n)}\}$ que converge uniformemente a un elemento $f \in F$.

  Probemos que la sucesión original converge a $f$. Dado $\epsilon > 0$ arbitrario,
  por ser $\{f_{\sigma(n)}\}$ convergente, existe $m_0 \in \mathbb{N}$ cumpliendo
  $$\forall n \geq m_0, \hspace{2mm} \|f_{\sigma(n)} - f\| < \frac{\epsilon}{2}$$

  Sabemos que $\{f_n\}$ es de Cauchy, como $\sigma(n) \geq n$, se tiene
  $$\|f_{\sigma(n)} - f_n\| < \frac{\epsilon}{2}$$

  siempre que $n \geq m$. Por tanto, dado $n \in \mathbb{N}$ cumpliendo $n \geq max\{m, m_0\}$,
  $$\|f_n - f\| \leq \|f_n - f_{\sigma(n)}\| + \|f_{\sigma(n)} - f\| <
  \frac{\epsilon}{2} + \frac{\epsilon}{2} = \epsilon$$

  luego $\{f_n\}$ es convergente como queríamos.
\end{flushleft}

\end{document}
