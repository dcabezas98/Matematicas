\documentclass{article}
\usepackage[left=3cm,right=3cm,top=2cm,bottom=2cm]{geometry} % page settings

\usepackage{amsmath,amssymb,amsthm,amsfonts}

\usepackage[spanish]{babel}

\usepackage{multirow}

\usepackage[utf8]{inputenc}
\setlength{\parindent}{0mm}

\usepackage[parfill]{parskip}

% Para el código
\usepackage{listings}
\usepackage{xcolor}
\definecolor{gray}{rgb}{0.5,0.5,0.5}
\newcommand{\n}[1]{{\color{gray}#1}}
\lstset{numbers=left,numberstyle=\small\color{gray}}

% Entorno para estilo de ejercicios
\newenvironment{ejercicio}[1]{\textbf{#1} \vspace*{5mm}}{\vspace*{5mm}}
\setlength{\parindent}{10pt} 

\begin{document}

\title{\Large max\{$f$,$g$\} y min\{$f$,$g$\} son medibles}
\author{David Cabezas}
\date{}
\maketitle

\begin{flushleft}
  \textbf{Sea $\Omega$ un espacio medible y
    $f,g:\Omega \longrightarrow \mathbb{R}$ funciones
    medibles. Consideramos
    $\alpha,\beta:\Omega \longrightarrow \mathbb{R}$ dadas por}
  $\alpha(\omega) = \mbox{min}\{f(\omega),g(\omega)\}, \ \beta(\omega)
  = \mbox{max}\{f(\omega),g(\omega)\} \hspace{2mm} \forall \omega \in
  \Omega.$

  \textbf{Probar que $\alpha$ y $\beta$ son medibles.}
\end{flushleft}  

\begin{flushleft}
  Dados $x,y \in \mathbb{R}$, tenemos
  $\mbox{max}\{x,y\}=\dfrac{x+y+|x-y|}{2}$, \
  $\mbox{min}\{x,y\}=\dfrac{x+y-|x-y|}{2}$

  Esto puede comprobarse fácilmente teniendo en cuenta que
  \[|x-y| =
\begin{cases}
x-y & \mbox{ si } x \leq y \\
y-x & \mbox{ si } y \leq x
\end{cases}
\]
Del mismo modo, tendremos
\[
  \alpha(\omega)=\mbox{min}\{f(\omega),g(\omega)\}=\dfrac{f(\omega)+g(\omega)-|f(\omega)-g(\omega)|}{2} \hspace{4mm} \forall \omega \in \Omega
\]
\[
  \beta(\omega)=\mbox{max}\{f(\omega),g(\omega)\}=\dfrac{f(\omega)+g(\omega)+|f(\omega)-g(\omega)|}{2} \hspace{4mm} \forall \omega \in \Omega
\]

Como $g$ es medible, $\lambda g$ lo es ($\lambda \in \mathbb{C}$), por
lo que $-g$ es medible. Además sabemos que la suma de funciones
medibles es medible, luego tenemos que $f-g$ es medible.

Por otra parte, la función valor absoluto es continua, por lo que al
componerla con $f-g$ obtenemos que $|f-g|$ es medible.

Queda comprobado que tanto $\alpha$ como $\beta$ son medibles, ya que
son resultado de sumar, restar y multiplicar por escalares funciones
medibles.

\hfill\qedsymbol
\end{flushleft}

\end{document}