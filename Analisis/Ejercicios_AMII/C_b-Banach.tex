\documentclass{article}
\usepackage[left=3cm,right=3cm,top=2cm,bottom=2cm]{geometry} % page settings
\usepackage{amsmath} % provides many mathematical environments & tools
\usepackage{amssymb}
\usepackage{amsfonts}
\usepackage[spanish]{babel}



\usepackage{multirow}

\usepackage{algorithm}
\usepackage{algpseudocode}
\usepackage{pifont}

\usepackage[utf8]{inputenc}
\setlength{\parindent}{0mm}

\usepackage[parfill]{parskip}

% Para el código
\usepackage{listings}
\usepackage{xcolor}
\definecolor{gray}{rgb}{0.5,0.5,0.5}
\newcommand{\n}[1]{{\color{gray}#1}}
\lstset{numbers=left,numberstyle=\small\color{gray}}

% Entorno para estilo de ejercicios
\newenvironment{ejercicio}[1]{\textbf{#1} \vspace*{5mm}}{\vspace*{5mm}}
\setlength{\parindent}{10pt} 

\begin{document}

\title{Probar que $(\mathcal{C}_b(A,\mathbb{R}), \|\cdot\|_\infty)$ es
  un espacio de Banach}
\author{David Cabezas} \date{}
\maketitle
\begin{flushleft}
  \textbf {Sea $A$ un subconjunto cualquiera de $R^N$. Probar que el
    espacio normado de funciones continuas y acotadas
    $(\mathcal{C}_b(A,\mathbb{R}), \|\cdot\|_\infty)$ es de Banach, es
    decir, completo.}

  En adelante $\|\cdot\|$ denota $\|\cdot\|_\infty$.

  Tomamos una sucesión de Cauchy $\{f_n\}$ de elementos del espacio.
  Dado un $\epsilon>0$, existe un $m \in \mathbb{N}$ tal que
  $$\forall p,q \in \mathbb{N} \text{ cumpliendo }  p,q \geq m \text{ se tiene } \|f_p - f_q\| < \epsilon$$
  en particular
  $$\forall n \in \mathbb{N} \text{ cumpliendo } n \geq m \text{ se
    tiene } \|f_n - f_m\| < \epsilon$$

  Notemos que
  $$\forall x \in A, |f_p(x)-f_q(x)| \leq \|f_p-f_q\| < \epsilon$$

  por lo que fijado un $x \in A$, la sucesión real $\{f_n(x)\}$ es de
  Cauchy y por tanto convergente. Esto nos da convergencia puntual
  para $\{f_n\}$ y un único candidato a límite, la función $f$ que
  cumple $f(x) = \lim\limits_{n \to \infty}f_n(x)$.

  Supongamos que la sucesión no converge uniformemente a $f$, esto es:
  $\exists \epsilon_0 > 0$ tal que
  $$\forall m \in \mathbb{N} \hspace{2mm} \exists n \geq m: \hspace{2mm} \exists x \in A
  \text{ cumpliendo } |f_n(x) - f(x)| \geq \epsilon_0$$

  Fijado este $\epsilon_0$, el hecho de que $\{f_n\}$ sea de Cauchy
  nos da un $m \in \mathbb{N}$ tal que
  $$\forall p, q \in \mathbb{N} \text{ cumpliendo }  p,q \geq m
  \text{ se tiene } \|f_p - f_q\| < \frac{\epsilon_0}{2}$$ luego se
  tendrá
  $$|f_p(x) - f_q(x)| < \frac{\epsilon_0}{2} \hspace{2mm} \forall x
  \in A$$

  Usando la suposición de que no $f$ converge uniformemente con este
  $m$
  $$\exists k \geq m, \hspace{2mm} \exists x_0 \in A \text{ tal que } |f_k(x_0) - f(x_0)| \geq \epsilon_0$$

  Lleguemos a un absurdo negando la convergencia puntual de $\{f_n\}$,
  $\forall n \in \mathbb{N}$ cumpliendo $n \geq m$ se tiene
  $$|f_n(x_0) - f(x_0)| = |f_n(x_0) - f_k(x_0) + f_k(x_0) - f(x_0)| \geq
  |f_k(x_0) - f(x_0)| - |f_n(x_0) - f_k(x_0)| > \epsilon_0 -
\frac{\epsilon_0}{2} = \frac{\epsilon_0}{2}$$

Entonces $\{f_n\}$ no converge puntualmente a $f$, lo cual es absurdo.
Por tanto $\{f_n\}$ debe converger uniformemente a $f$. 
\end{flushleft}

\begin{flushleft}
  Como cada función $f_n$ de la sucesión es continua, sabemos que $f$
  lo es. Para probar que $f$ es acotada procedemos del siguiente modo:
  fijamos $\epsilon = 1$, $\exists m \in \mathbb{N}$ tal que
  $|f_m(x) - f(x)| < 1$ $\forall x \in A$, por tanto
  $|f(x)| < |f_m(x)| + 1$ $\forall x \in A$, luego $f$ está acotada
  por estarlo $f_m$, que es un elemento del espacio.
\end{flushleft}

\begin{flushleft}
  Esto prueba que toda sucesión de Cauchy de elementos del espacio
  converge uniformemente a un elemento del espacio, luego es de Banach.
\end{flushleft}

\end{document}
