% This text is proprietary.
% It's a part of presentation made by myself.
% It may not used commercial.
% The noncommercial use such as private and study is free
% Dec 2007
% Author: Sascha Frank 
% University Freiburg 
% www.informatik.uni-freiburg.de/~frank/
%

\documentclass{beamer}
\usetheme{Warsaw}
\setbeamertemplate{navigation symbols}{}
%gets rid of bottom navigation bars
\setbeamertemplate{footline}[frame number]

\usepackage[english,spanish]{babel}

\usepackage{floatrow}
\usepackage{graphicx}
\usepackage{caption}

\usepackage{amssymb}

\usepackage{ bbm }

\usepackage{subfigure}

\usepackage{enumerate}

\usepackage{tikz}
\usetikzlibrary{calc, shapes, backgrounds, automata, positioning}
\tikzset{
	every point/.style = {radius={\pgflinewidth}, opacity=1, draw, solid, fill=white},
	pt/.pic = {
		\begin{pgfonlayer}{foreground}
			\path[every point, #1] circle;
		\end{pgfonlayer}
	},
	point/.style={insert path={pic{pt={#1}}}}, point/.default={},
	colored point/.style = {point={fill=#1}},
	point name/.style = {insert path={coordinate (#1)}}
}

\beamersetuncovermixins{\opaqueness<1>{10}}{\opaqueness<2->{5}}

\AtBeginSection[]
{
	\begin{frame}<beamer>{Índice}
		\tableofcontents[currentsection]
	\end{frame}
}

\makeatletter
\AtBeginPart{%
	\beamer@tocsectionnumber=0\relax
	\setcounter{section}{0}
	\frame{\partpage}%
}
\makeatother

\newenvironment{redblock}[1]{%
	\setbeamercolor{block body}{bg=red!10, fg=black}
	\setbeamercolor{block title}{bg=red!70!black, fg=white}
	\begin{block}{#1}}{\end{block}}

\newenvironment{greenblock}[1]{%
	\setbeamercolor{block body}{bg=green!10, fg=black}
	\setbeamercolor{block title}{bg=green!60!black, fg=white}
	\begin{block}{#1}}{\end{block}}

\newenvironment{cyanblock}[1]{%
	\setbeamercolor{block body}{bg=cyan!10, fg=black}
	\setbeamercolor{block title}{bg=cyan!50, fg=white}
	\begin{block}{#1}}{\end{block}}

\newcommand{\co}[1]{\operatorname{co}(#1)}
\newcommand{\cloco}[1]{\overline{\operatorname{co}}(#1)}

\begin{document}
\title{El Teorema de Russo-Dye y refinamientos}  
\author{David Cabezas Berrido}
\date{\vspace{-3mm}} 

\begin{frame}
	~\vspace{-10mm}
\titlepage
\vspace{-10mm}

\begin{center}
Trabajo final de la asignatura \\ \smallskip Métodos Avanzados de Análisis Funcional y Análisis de Fourier
\end{center}
\end{frame}

\begin{frame}\frametitle{Índice}\tableofcontents
\end{frame} 

\section{Introducción}

\begin{frame}
		
\begin{greenblock}{R.R. Phelps en ``Extreme points in function algebras'' (1965)}
	¿De qué formas se puede expresar un elemento de una C*-álgebra como combinación convexa de elementos unitarios?
\end{greenblock}

\pause

\begin{block}{R.R. Phelps en ``Extreme points in function algebras'' (1965)}
	En una C*-álgebra conmutativa y unital, la envolvente convexa de los elementos unitarios es densa en la bola cerrada unidad.
\end{block}

\pause

\begin{redblock}{Teorema de Russo-Dye (1966)}
	En una C*-álgebra unital, la envolvente convexa de los elementos unitarios es densa en la bola cerrada unidad.
\end{redblock}

\pause

\begin{itemize}
	\item Manifiesta la abundancia de elementos unitarios en una C*-álgebra (unital). \pause
	\item Posteriormente refinado y versionado.
\end{itemize}

\end{frame}

\section{Preliminares}

\subsection*{Notación}
\begin{frame}\frametitle{Notación}
	\begin{itemize}
		\setlength\itemsep{0.5cm}
		\item $A$ C*-álgebra unital sobre el cuerpo $\mathbb{C}$.
		\item $\mathbbm{1}\in A$ su unidad. \pause
		\item $A_1=\{a\in A: \|a\|<1\}$	bola abierta unidad de $A$.
		\item $B_A=\{a\in A: \|a\|\leq 1\}$ bola cerrada unidad de $A$. \pause
		\item $U=\mathcal{U}(A)$ grupo de elementos unitarios de $A$.
	\end{itemize}
\end{frame}

\subsection{Descomposición polar de elementos invertibles}

\begin{frame}\frametitle{Descomposición polar de invertibles}
\begin{block}{Lema}
	Todo elemento invertible $a\in A$ admite una \emph{descomposición polar} de la forma $a=u|a|$, donde $|a|=(a^*a)^{1/2}$ es positivo y $u$ es unitario.
\end{block}
\pause
\begin{cyanblock}{Demostración}
{\small
	$|a|^2:=a^*a\in A$ invertible y positivo, admite raíz cuadrada $|a|$ (CFC). \pause
	\vspace{-2mm}\begin{gather*}
	|a|(|a||a|^{-2})=(|a||a|)|a|^{-2}=|a|^2|a|^{-2}=\mathbbm{1}\\
	(|a|^{-2}|a|)|a|=|a|^{-2}(|a||a|)=|a|^{-2}|a|^2=\mathbbm{1},
	\end{gather*}\vspace{-6mm}
	
	$|a|$ invertible con $|a|^{-1}=|a||a|^{-2}=|a|^{-2}|a|\Rightarrow |a|^{-1}|a|^{-1}=|a|^{-2}$.
}
\end{cyanblock}

\end{frame}

\begin{frame}\frametitle{Descomposición polar de invertibles}
	\begin{block}{Lema}
		Todo elemento invertible $a\in A$ admite una \emph{descomposición polar} de la forma $a=u|a|$, donde $|a|=(a^*a)^{1/2}$ es positivo y $u$ es unitario.
	\end{block}
	\begin{cyanblock}{Demostración}
		{\small
			$u:=a|a|^{-1}\in A$ invertible. \pause Como $|a|^{-1}$ auto-adjunto,
			\vspace{-2mm}\begin{gather*}
			uu^*=a|a|^{-1}|a|^{-1}a^*=a|a|^{-2}a^*=a(a^*a)^{-1}a^*=aa^{-1}(a^*)^{-1}a^*=\mathbbm{1}\\
			u^*u=|a|^{-1}a^*a|a|^{-1}=|a|^{-1}|a|^2|a|^{-1}=|a|^{-1}|a||a||a|^{-1}=\mathbbm{1},
			\end{gather*}\vspace{-6mm}
			
		$u$ unitario. \hfill$\square$
		}
	\end{cyanblock}
	
\end{frame}

\section{Teorema de Russo-Dye}

\subsection*{Resultado}
\begin{frame}\frametitle{El resultado}

\begin{redblock}{Teorema de Russo-Dye (1966)}
Sea $A$ una C*-algebra unital y $U=\mathcal{U}(A)$ su grupo de unitarios. Entonces, $B_A=\cloco{U}$.
\end{redblock}

\pause

Original: ``A note in unitary operators in C$^*$-algebras'' (A.H. Dye \& B. Russo, 1966). \\ \vspace{2mm} \pause

Prueba elemental: ``Shorter Notes: An Elementary Proof of the Russo-Dye Theorem'' (L.T. Gardner, 1984).

\end{frame}

\subsection*{Demostración}
\begin{frame}%\frametitle{Demostración}
	
	\begin{cyanblock}{Demostración}
		{\small
			Basta probar $A_1\subset\cloco{U}$. \pause Tomamos $x\in A_1$, $u\in U$ cualesquiera, sea $y:=\dfrac{x+u}{2}=\dfrac{xu^*+\mathbbm{1}}{2}u$. \pause
			$\|xu^*\|\leq \|x\|\|u^*\|=\|x\|<1\Rightarrow xu^*+\mathbbm{1}$ invertible (también $y$). \pause Podemos escribir $y=v|y|$ con $v\in U$ y $|y|=(y^*y)^{1/2}\in B_A$ positivo.\vspace{2mm} \\ \pause
			
			$\||y|^2\|=\|y^*y\|=\|y\|^2<1\Rightarrow|y|^2\leq \||y|^2\|\mathbbm{1}\leq \mathbbm{1}\Rightarrow\mathbbm{1}-|y|^2\geq 0$
			admite raíz cuadrada. \pause Por tanto, $|y|=(w+w^*)/2$, donde \vspace{-2mm}
			\[w=|y|+i(\mathbbm{1}-|y|^2)^{1/2},\quad w^*=|y|-i(\mathbbm{1}-|y|^2)^{1/2}.\]\vspace{-7mm} \pause
			
			Además, $w,w^*=w^{-1}$ unitarios, ya que $|y|$ y $\mathbbm{1}-|y|^2$ conmutan.\vspace{1mm} \pause
			
			\textcolor{red}{Hemos probado $x+u=vw+vw^*\Rightarrow A_1+U\subset U+U$ ($*$).} \vspace{1mm} \pause
			
			{\scriptsize $\dfrac{x+U}{2}\subset\co{U}\Leftrightarrow U\subset 2\co{U}-x\Rightarrow \co{U}\subset 2\co{U}-x\Rightarrow\dfrac{x+\co{U}}{2}\subset \co{U}$} \vspace{1mm} \pause
			 
			Por tanto, la sucesión $x_0=u$ y $x_{n+1}=(x+x_n)/2$ yace en $\co{U}$. Claramente $x_n\to x$. \hfill$\square$
		}
	\end{cyanblock}
	
\end{frame}

\begin{frame}%\frametitle{Demostración}
	
	\begin{block}{Observación de C.K. Fong: $A_1\subset \co{U}$}
		\pause
		Como $x\in A_1$, podemos tomar $x'\in A_1$ tal que $x\in [u,x'[$. \\ \pause La sucesión $x_0=u$ y $x_{n+1}=(x'+x_n)/2$ cumple $x\in [u,x_n]$ para $n$ lo bastante grande, luego $x\in\co{U}$.	
	\end{block}

	\pause

	\vspace{4mm}

	Claramente, $A=\operatorname{span}U$.
\end{frame}

\subsection{Una importante consecuencia}

\begin{frame}\frametitle{Aplicación}
	
	\begin{block}{Corolario}
		Sea $A$ una C*-álgebra unital y $U=\mathcal{U}(A)$ su grupo de unitarios, y sea $X$ un espacio normado arbitrario. Entonces, una aplicación lineal $\phi:A\rightarrow X$ es continua si y solo si $\phi$ está acotada en $U$. Además, se tiene la siguiente igualdad: \vspace{-2mm}
		\begin{equation*}\label{eq:norm-unitary}
		\|\phi\|=\sup_{u\in U}\|\phi(u)\|.
		\end{equation*}
	\end{block}

\pause

\begin{cyanblock}{Demostración}
	{\small
		Definimos una nueva norma en $A$ \vspace{-2mm}
		\[\|a\|_U:=\inf\left\{\sum_{j=1}^n|\lambda_j|:a=\sum_{j=1}^n \lambda_j u_j,\ \lambda_j\in\mathbb{K},\ u_j\in U\right\}\] \vspace{-4mm} \pause
		
		Cumple $\|a\|\leq \|a\|_U\ \forall a\in A$. Además, $a\in \co{U}$ implica $\|a\|_U\leq 1$.
	}
\end{cyanblock}
\end{frame}

\begin{frame}%\frametitle{Aplicación}
	
	\begin{cyanblock}{Demostración}
		{\small			
			Para cada $\varepsilon>0$, $b=\dfrac{a}{\|a\|+\varepsilon}\in A_1\subset\co{U}$, luego $\|b\|_U\leq 1$ y $\|a\|_U\leq \|a\|+\varepsilon$. \pause Por tanto, $\|a\|=\|a\|_U \ \forall a\in A$. \vspace{1mm} \pause
			
			Sea $K=\sup_{u\in U}\|\phi(u)\|\in\mathbb{R}^+_0$. \pause Para cada $a=\sum_{j=1}^n \lambda_j u_j$ (con cada $\lambda_j\in\mathbb{K}$ y $u_j\in U$) tenemos\vspace{-3mm}
			\[\|\phi(a)\|=\bigg\|\sum_{j=1}^n\lambda_j\phi(u_j)\bigg\|\leq \sum_{j=1}^n|\lambda_j|\|\phi(u_j)\|\leq K\sum_{j=1}^n|\lambda_j|,\] \vspace{-4mm} \pause
			
			luego $\|\phi(a)\|\leq K\|a\|_U=K\|a\|$. \pause Por tanto, $\phi$ es continua con $\|\phi\|\leq K$. \vspace{1mm} \pause
			
			Por la definición de $K$, existe una sucesión $\{u_n\}$ en $U\subset B_A$ tal que $\|\phi(u_n)\|\to K$, de modo que $\|\phi\|\geq K$.  \hfill$\square$
		}
	\end{cyanblock}
	
\end{frame}

\section{Refinamiento de Kadison y Pedersen}

\subsection*{Primer Refinamiento}
\begin{frame}\frametitle{Mejora del teorema}
	
	Gardner prueba \textcolor{red}{$A_1+U\subset U+U$ ($*$)}
	
	\pause
	
	\begin{redblock}{Teorema}
		Sea $A$ una C*-algebra unital y $U=\mathcal{U}(A)$ su grupo de unitarios. Si un elemento $a\in A$ cumple $\|a\|<1-\frac{2}{n}$ para algún entero $n>2$, entonces existen $n$ elementos unitarios $u_1,\ldots,u_n\in U$ tales que $a=n^{-1}(u_1+\cdots+u_n)$.
	\end{redblock}
	
	\pause
	
	``Means and convex combinations of unitary operators'' (R.V. Kadison \& G.K. Pedersen, 1985).	
\end{frame}

\begin{frame}\frametitle{Demostración}
	
	\begin{cyanblock}{Demostración}
		{\small			
		Sean $x\in A_1$ and $u\in U$ cualesquiera, consideremos el elemento 
		\vspace{-2mm}\[z=u+(n-1)x=u+x+(n-2)x\in A.\]\vspace{-6mm}
		
		\pause Existen $u_1,v_1\in U$ tales que $u+x=u_1+v_1$. \pause Luego	
		\vspace{-2mm}\[z=u_1+v_1+(n-2)x=u_1+v_1+x+(n-3)x.\]\vspace{-6mm}
		
		\pause Existen $u_2,v_2\in U$ tales que $v_1+x=u_2+v_2$. \pause Luego
		\vspace{-2mm}\[z=u_1+u_2+v_2+(n-3)x=u_1+u_2+v_2+x+(n-4)x.\]\vspace{-6mm}
				
		\pause En $n-3$ pasos más, \vspace{-4mm}
		\begin{equation}\label{eq: sum of unitaries}
		z=u+(n-1)x=\sum_{j=1}^n u_j,\hspace{2mm}\text{donde $u_j\in U$ para cada $j=1,\ldots,n$}
		\end{equation}
		}
	\end{cyanblock}
	
\end{frame}

\begin{frame}\frametitle{Demostración}
	
	\begin{cyanblock}{Demostración}
		{\small ~\vspace{-4mm}
			\begin{equation*}
			z=u+(n-1)x=\sum_{j=1}^n u_j,\hspace{2mm}\text{donde $u_j\in U$ para cada $j=1,\ldots,n$} \tag{\ref{eq: sum of unitaries}}
			\end{equation*} \vspace{-5mm}
					
			\pause Notemos ahora que \vspace{-2mm}
			\begin{align*}
			\|(n-1)^{-1}&(na-\mathbbm{1})\|\leq (n-1)^{-1}(n\|a\|+1) \\ <&(n-1)^{-1}\big(n(1-2/n)+1\big)=(n-1)^{-1}(n-1)=1,
			\end{align*}\vspace{-6mm}
			
			\pause luego $(n-1)^{-1}(na-\mathbbm{1})\in A_1$. \vspace{2mm} \\ \pause Aplicamos \eqref{eq: sum of unitaries} con $x=(n-1)^{-1}(na-\mathbbm{1})$ y $u=\mathbbm{1}$:\vspace{-4mm}
			\[u+(n-1)x=\mathbbm{1}+(na-\mathbbm{1})=na=u_1+u_2+\cdots+u_n=\sum_{j=1}^n u_j\]\vspace{-6mm}
			
			\hfill$\square$
		}
	\end{cyanblock}
\end{frame}

\subsection*{Consecuencia}

\begin{frame}\frametitle{Aplicación}
	
	\begin{block}{Corolario}
		Todo elemento de una C*-algebra unital $A$ es un multiplo positivo de la suma de tres unitarios.	
	\end{block}
	
	\pause
	
	\begin{cyanblock}{Demostración}
		{\small
			Dado cualquier $a\in A$, tomamos $\varepsilon>0$ cualquiera y consideramos el elemento $b=\dfrac{1}{3(\|a\|+\varepsilon)}a$. \pause Claramente $\|b\|<1/3$. \vspace{2mm} \\ \pause
			
			Por el teorema anterior con $n=3$, podemos escribir $b=(u_1+u_2+u_3)/3$ con $u_j\in U$ para $j=1,2,3$. \pause Por tanto, \vspace{-2mm}\[a=3(\|a\|+\varepsilon)b=(\|a\|+\varepsilon)(u_1+u_2+u_3)\]\vspace{-10mm}
			
			\hfill$\square$			
		}
	\end{cyanblock}
\end{frame}

\subsection*{Segundo Refinamiento}

\begin{frame}\frametitle{Otra vuelta de tuerca}
	
	\begin{redblock}{Teorema}
		Sea $A$ una C*-algebra unital y $U=\mathcal{U}(A)$ su grupo de unitarios. Si un elemento $a\in A$ cumple $\|a\|\leq 1-2/n$ para algún entero $n>2$, entonces existen $n$ elementos unitarios $u_1,\ldots,u_n\in U$ tales que $a=n^{-1}(u_1+\cdots+u_n)$.
	\end{redblock}
	
	\pause
	
	``Means of unitary operators, revisited'' (U. Haagerup, R.V. Kadison \& G.K. Pedersen, 2007).	
\end{frame}

%%%%%%%%%%%%%%%%%%%%%%%%%%%%% FIN

\begin{frame}[plain,c]
	\begin{center}
		\Huge Fin.
	\end{center}
	\vspace{10mm}
	\begin{center}
		\huge Gracias por la atención.
	\end{center}
\end{frame}

\end{document}